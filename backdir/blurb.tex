\documentclass[12pt]{article}
\usepackage{epsfig}

\oddsidemargin  0pt
\evensidemargin 0pt
\marginparwidth 0pt
\marginparsep   0pt
\topmargin      0pt
\headheight     0pt
\headsep        0pt
\textheight     9.0in
\textwidth      6.5in



\begin{document}

\centerline{\bf Toward Human-Level Cognitive Adequacy}


{\em Our long-range aim is to design and implement common sense in a computer.}
Common sense in a computer is a bit hard to define, but the idea we
are aiming at is comparable to human-level common sense (often
understood as distinct from expert or special cleverness).  For
instance, solving the mutilated checkerboard problem takes a special
clever insight, and thus is {\em not} what we have in mind. But note
that much of the AI community would not draw the definitional lines as
we have; for that reason, we sometimes use the expression ``cognitive
adequacy'' to refer to our conception.  This is intended to suggest a
kind of general-purpose reasoning ability that will serve the agent to
``get along'' (learning as it goes) in a wide and unpredicted range of
environments.  Consequently, one hallmark of common sense is the
ability to recognize, and initiate appropriate responses to, novelty,
error, and confusion.  Examples of such responses include learning
from mistakes, aligning action with reasoning and vice versa, and
seeking (and taking) advice.

A closely related hallmark is the ability to reason about anything whatever
(that is brought to one's attention). This does not mean being
clever about it, or knowing much about it, or being able to
draw significant conclusions; it can mean as little as realizing that
the topic is not understood, asking for more information, and learning
appropriately from whatever advice is given.  That may seem like very
little, if our model is to have clever solutions to tricky problems. But
consider this: virtually no AI programs exhibit even that ``little''
amount of elementary common sense; they are not able to know when they
are confused, let alone seek -- and use -- clarifying data. On the
other hand, cleverness -- in highly limited domains and for tightly
specified representations -- has been built into many programs, a kind
of ``idiot savantry'' that fails utterly when outside those narrow
strictures.

One large piece of what is needed for cognitive adequacy, then, is
what we call ``perturbation tolerance'': the ability to keep going
adequately when subjected to unanticipated changes.  This includes
changes to the knowledge base (KB); e.g.~the
changes might introduce inconsistencies, or make a goal impossible or
ambiguous. Worse, the knowledge representation (KR) system might
change (new terms, new meanings for old terms, different notational
conventions, etc), especially if other agents are involved; and of
course there are typos (missing parentheses and the like) that appear
to defy any prearranged methodology.  And there are
changes to physical sensors and effectors, and how things in
the world work.

Then what is it to ``keep going adequately'' in the face of such
changes?  Among other things, this will require (i) never ``hanging''
or ``breaking''; (ii) recognizing when there is a difficulty to be
addressed; (iii) making an assessment of options to deal with the
difficulty; (iv) choosing and putting an option into action.  Such a
suite of abilities will require keeping track of one's own history of
activity, including one's own past reasoning.  Such an agent will then,
in Nilsson's phrase, have a {\em lifetime of its own}, and keeping
track of its own processes and history will allow it to look back at
what it is doing and use that knowledge to guide its upcoming
behavior.

People tend to do this well.
%Is this ability an evolutionary hodgepodge, a holistic
%amalgam of countless parts with little or no intelligible structure?
%Or might there be some few key modular features that provide this
%``adequacy''? 
We think there is strong evidence as to how, and we have a specific
hypothesis about it and how to build it in a computer.  In a nutshell,
we propose what we call the {\em metacognitive loop}---that allows a
reasoning agent to note, and attempt to correct, errors in real
time---as the essential distinguishing feature of cognitive adequacy,
including perturbation tolerance. And we claim that the state of the
art is very nearly where it needs to be, to allow this to be designed
and implemented.

\end{document}

