\documentclass{amsart}
%\documentclass[fullpage]{article}

%\centerline{Don Perlis}

\title{\bf Theory and Application of Self-Reference:\\Logic and Beyond}

\author{Don Perlis}

% \author{Don Perlis\\
%         University of Maryland\\
%         College Park MD 20742 USA\\
%         perlis@cs.umd.edu\\
%         www.umd.edu/~perlis\\
%         301-405-2685
%         fax: 301-405-6707}


\begin{document}


\begin{abstract}
This paper begins with a brief (and probably rather personal and
one-sided) overview of self-reference as seen in logic and AI, then
slides over to speculations on self-reference in natural language,
and eventually returns to AI and even more speculative theories of the
conscious mind based on one kind of self-reference (arguably the most
fundamental kind). 
\end{abstract}

\maketitle

%\begin{verbatim}

\section{Introduction}

kjds this isn  jeej o ksjd

Self-reference has been a topic of study in formal philosophy and
logic for a great many years, indeed for millenia.  Yet, I shall
argue, many ostensible cases of self-reference so studied are not, in
fact, cases of reference at all. To get to reference, and to
self-reference, one must go beyond the usual cases, formal and
otherwise. To present this argument in context,
I will begin with a brief review of some of the more familiar examples
of (so-called) self-reference and the associated puzzles and problems
(``conundra'') that have drawn attention to them.  I will then discuss
a particular solution to these conundra, based on work of Gilmore and
of Kripke, a solution that seems to avail itself of a ``normal order
principle''.  This in turn will lead to the question as to how
satisfactory such a principle is, in light of further examples from
natural language and from cognitive philosophy. Finally, some
tentative conclusions are drawn, about the nature of language,
reference, and self.


\section{Some conundra}
Perhaps the most famous example of a self-referential utterance is
the so-called $Liar$:

$$This~ sentence~ is~ false.$$
\smallskip
\noindent
The Liar, or $L$ for short, has two curious features: (i)
it appears to refer to itself, and (ii) it appears to contradict
itself. The latter feature is the one that has received the vast bulk
of attention, and we shall give that aspect attention as well.

However, we also will critically examine the former feature, in
stages, paying special attention to the word ``This''.  At this early
stage let us simply note that there is an issue as to what, if
anything, can guarantee that ``This'' in $L$ succeeds in referring to
$L$ itself, as opposed (say) to some other sentence that may have
recently been uttered or pointed to.  An altered ``display'' version
is 
$$DL: \hspace{1in} DL~ is ~false.$$
\smallskip
\noindent
But what guarantees that the name ``DL'' in the display refers to
the sentence that follows it? Presumably it is our agreement that it
so refer; but then there is personal agency involved.

Perhaps ``This sentence that I 
am now writing/uttering/expressing, beginning with the word $This$, is
false'' will provide a more convincing guarantee, wearing, so to
speak, its meaning on its sleeve. But this then rests upon a clear
reference for ``I'', presumably once again an agent with referential
{\em intentions}, 
a matter seemingly far removed from the usual concerns in discussions
of the Liar. Another sort of guarantee might be along the lines of ``The
only sentence written on the whiteboard at 4:19pm on May 12, 2003, in
room 3259 of the A. V. Williams Building at the University of
Maryland, is false.'' We will set these new concerns aside for now,
returning to them nearer the end of our essay.

The usual concern with $L$ is that it appears to contradict itself: If
$L$ is true, then what $L$ says is true, i.e., $L$ is false, so $L$
cannot be true.  But if $L$ is not true, then $L$ is false, and this
is what $L$ says, so $L$ is true after all.  Thus it appears that $L$
can have no consistent truth-status.  Yet then, in particular, $L$ is
not true, and so what it says again seems true.

Perhaps ``not true'' and ``false'' should be distinguished.  Yet ``not
true'' is what ``false'' means, by common accounts.  And in any case,
we can recast the Liar as sentence $L'$:
$$This ~sentence~ is ~not ~true.$$
\smallskip
\noindent
Now the distinction between ``false'' and ``not true'' is irrevelant,
and yet the conundrum remains.


One natural-enough reaction to the above circumstances is to
suppose that some sentences are, after all, neither true nor false,
indeed neither true nor not true.  This may seem to play fast and
loose with the word ``not''---after all, ``not'' simply asserts the
failure of what follows it.  But perhaps there is something hidden
here.  In order to be a candidate for failure in the matter of its
truth, and sentence must first have a potential truth that can fail,
i.e., it must have a clear enough meaning that can be
measured against some criterion of truth. How does a sentence acquire
a meaning? This is the subject of much dispute, and we will return to
it a bit later on. But what we need now seems more modest: the notion
that meaning---whatever it is---allows a distinction between
sentences $S$ for which there is a clear separation between the
meanings of 
\smallskip
$$S ~ is ~ true.$$
and
$$S ~ is ~ not ~ true.$$
\smallskip
and those for which there is not such a distinction.

Not all sentences meet 
this (admittedly imprecise) ``separation criterion''.  The
following examples may help clarify the point:
$$Colorless~ green~ ideas~ sleep~ furiously.$$
$$That~ sentence~ is ~ false.$$
$$This~ sentence~ no ~verb.$$
\smallskip
\noindent
The first of these is a famous example due to Chomsky; it illustrates
a grammatically (syntactically) correct English sentence, but one that
has little if any meaning (semantic nonsense): one would be
hard-pressed to say what it means, and therefore whether it is true or
not.  The second example
seems unproblematically meaningful, yet without further information as to
what ``That'' refers to, it cannot be said to be either true or
false. The third example is not a sentence, strictly speaking, but it
has an emphatically obvious meaning, and indeed one that is true.

The present point is that, as in the second example above, a sentence
may be neither true nor not true, in the sense that its meaning does
not make a clear enough separation between these. In the case of the second
example, the lack of separation appears to rest on missing external
information (what ``That'' refers to).  The Liar, on the other hand
(assuming we take its ``This'' as unproblematically self-referential)
fails to make a clear enough separation due to an internal
circularity.  In that respect the Liar is much like the $TruthTeller$:
$$This~ sentence~ is~ true.$$
\smallskip
\noindent
which (although not contradictory) also appears not to make a clear enough
claim to determine it to be true or not: its truth seems to presuppose
its truth, so we never manage to pin it down or separate it from its
failure.

This suggests a {\em Normal Order Principle}: Truth (of a sentence
$S$) is a relation between the world, the sentence $S$, and the
meaning $m$ of $S$, where the relation has a temporal character: $W$
precedes $S$ which in turn must precede $m$ (a relation between $S$
and the world), which in turn precedes the truth (or lack thereof) of
$S$.  This---also sometimes referred to as {\em grounding}---will be the
subject of a later section.  Here we shall instead briefly mention a
few other matters, formal and otherwise. 

The examples we have considered so far are informal, based largely on
commonsense notions. However, it is not hard to capture similar
behaviors in more formal dress.  A key component of the formalization
is the Diagonal Lemma, which asserts that in any reasonably expressive
formal theory $F$, for each unary wff $Px$ there is a sentence $p$ such that
it is provable in $F$ that $p \iff \neg P'p'$.  Here $'p'$ is
a name for $p$, i.e., a term that allows us to predicate $P$ of $p$.  Given
the Diagonal Lemma various results follow, some more easily than
others:

\begin{enumerate}
\item
$Schema~ T$ (which says that, for all wffs, $a$ $T'a' \iff a$)
is contradictory.  Here $T$ is to be thought of as a truth-predicate,
asserting the truth of its argument.  But applying the Diagonal
Lemma, we get, for a suitable wff $t$, $t \iff \neg T't'$ and so by
Schema $T$, $t \iff \neg t$, a contradiction. 
\item
From the above result, renaming $t$ as $L$, we have
$$L \iff \neg T'L'$$
a formal counterpart of the Liar: $L$ is equivalent to its own
denial. And as just seen, this leads to a formal inconsistency.
\item
With more work (G\"{o}del) one can devise a wff $Thm$ such that, for all
sentences $a$, $Thm 'a'$ is provable in theory $F$ iff and only if $a$
is provable in $F$; that is, $Thm$ behaves like a provability-predicate.  But by
the Diagonal Lemma we also have that, for some wff $g$, $g \iff \neg
Thm 'g'$; i.e., $g$ is equivalent to its own unprovability in $F$. It follows that
if $g$ is provable, so is $\neg g$. Hence we 
get G\"{o}del's Theorem: $g$ is either unprovable (and hence true, in
the sense that what it ``asserts''---its own unprovability---holds),
or $F$ is inconsistent.
\end{enumerate}

Another example brings us closer to artificial intelligence, where
what a reasoning agent can be said to know is of great interest.
Suppose there to be an very wise agent, $A$, who has excellent powers of
self observation, including knowledge of what $A$ does and does not
know.  That is,  $A$ knows a fact $f$ if and only if $A$ also knows
that $A$ knows $f$, and moreover, if $A$ does not know $f$ then $A$
knows that $A$ does not know $f$. Formally (but suppressing quote
marks for readability):

$$K_A(f) \iff K_A(K_A(f))$$
and
$$\neg K_A(f) \iff K_A(\neg K_A(f))$$

Suppose we also regard $A$'s knowledge as being what $A$ can $prove$,
so that, in effect, $A$ is a formal theory.  Then $K$ behaves like
a strong version of $Thm$, i.e., the above schemata become

$$A \vdash f \iff A \vdash K(f)$$
and
$$A \not \vdash f \iff A \vdash \neg K(f))$$
\smallskip
\noindent
But now we can show that $A$ is inconsistent.  

From the Diagonal Lemma we get a wff $k$
such that $\vdash k \iff \vdash \neg K(f)$. But from the above schemata,
if $k$ is not provable, then $\vdash
\neg K(f)$ and then $\vdash k$, i.e., $k$ is provable after all;
so it cannot be that $k$ is unprovable.  But then $k$ is
provable, and so $\vdash K(k)$, and also $\neg K(k)$; thus $A$ is
inconsistent.  This we may call the ``No perfect-self-knowledge
theorem''.  It indicates that artificial agents (as well as ``real''
agents) are subject to certain formal constraints on their knowledge.

All the above {\em formal} results, it should be admitted, are purely
syntactic, and reference (let alone self-reference) plays no real
role. For example, that $L$ refers to anything is irrelevant to the
above proof that the formal $Liar$ is inconsistent.  And the G\"odel
sentence $g$ does not really refer to anything at all, let alone to
its own unprovability.

And who cares? Isn't self-reference just a curiosity?---an accident
allowed by a side-effect of over-expressive language, with
surprisingly useful but equally accidental application in formal
logic, and of no deep significance in itself?  Cannot we then simply
ignore the bad (contradictory) cases (e.g., Schema $T$ which allows
the $Liar$ in full force) and welcome others (such as provided by
G\"odel)? Perhaps, but we shall argue otherwise.  First however we
shall look at some standard methods for isolating the bad cases: the
Tarski Hierarchy, and the Gilmore-Kripke partial models that appear to
provide a formal characterization of Normal Order.


\section{The Tarski Hierarchy}

Since the above formal results are just that, formal (syntactic) and
not dependent on semantics, then it might be possible to keep the
advantages of certain ``seeming'' self-reference (as in G\"{o}del's
powerful Theorem) without the disadvantages of the $Liar$ (such as
inconsistency). Tarski showed how to do this by means of a restriction
on how languages refer. He posited a hierarchy of languages $L_1, L_2,
\ldots$ where each $L_{j+1}$ has expressions that refer only to objects in
a previously defined language $L_{j}$.  There is then no expression
that can refer to itself, and no truth predicate $T$ that can be
applied to an expression that also has $T$ in it. Rather, each $L_j$
has its own truth-predicate $T_j$ that can be applied (only) to
expressions in $L_{j-1}$.  This approach simply banishes
self-reference from expression altogether, while leaving intact the
syntactic vestiges needed for the Diagonal Lemma (and useful formal
results such as G\"{o}del's Theorem).

Thus in particular, Schema $T$ is banished, and with it the
possibility of the $Liar$ and contradiction. On the other hand, the
Hierarchy seems to banish too much. There are perfectly innocuous but
semantically-based cases of self-reference, such as:
$$This~ sentence~ has~ five~ words.$$
\noindent
Yet this is not expressible in the Tarski Hierarchy.  Nor is the following pair
of straightforward sentences, each referring to the other (one happens
to be false):
$$The~ sentence~ below~ has~ seven~ words.$$
$$The~ sentence~ above~ has~ six~ words.$$

Note that the Hierarchy banishes statements that have a sentence $S$
appearing both as $S$ {\em simpliciter} and as a named object $'S'$
together, for then the latter would refer to the former, a violation
of the hierarchical order. But the Normal Order Principle mentioned
earlier does allow {\em expression} of such statements; it simply clamps down
on how a statement with a truth-predicate can itself be judged true or
false: namely, when there is sufficiently clear separation between these and,
further, when the truth-judgement is made {\em after} the meaning is
determined.

\section{The Gilmore-Kripke Approach}

Let us review the Normal Order Principle.  A paradigmatic case is
this: {\em Snow is white.} Here there is a fact of the matter (that
crystalline water tends to reflect fairly uniformly across the
spectrum of visible light) and there is also a sentence (the one just
cited); and there is a connection between them that allows (in
principle) its verification once the former ingredients are in place.
We have the world, then we have the sentence, and then we ascertain a
potential connection between them that distinguishes two cases (true,
false), and finally we judge one of these to obtain.  If instead one sentence
refers to another (rather than, say, to snow), then the referring
sentence treats the referent sentence as part of the world.   Put yet
another way: we look to see (e.g., that $x$), then record the result
$True(x)$.  The temporal order is this: first there is (i) the world $W$, next (ii) a
sentence $S$, then (iii) a connection of meaning between $S$ and $W$, and
finally (iv) a determination of truth or falsity with respect to that
meaning. If this process fails, then we have nevertheless learned
something about $S$, namely that it is neither true nor false---which
is simply to say again that the process has failed!  

In other words, the Normal Order
Principle defines truth and falsity as successful outcomes of this process.
As such it seems natural enough, at least at first
glance. And it has a formal counterpart in work of Gilmore
\cite{gilmore:consistency} and Kripke \cite{kripke:outline},
who (independently and differently) provided analyses of certain
conundra and tools for addressing them, along such lines.  Gilmore dealt with
set-theoretic versions of the $Liar$, such as {\em Russell's Paradox}:
is the set $R$ whose members are all those sets that do not contain themselves
as members, a member of itself?  His treatment focuses on the
set-membership relation $x \in y$ and a special method for
constructing models in partial steps, akin to the
Normal Order Principle. This leads to a consistent formulation 
in which nevertheless the Russell set can be defined, but the
question as to whether it is or is not a member of itself is, in
effect, not well-defined. That is, $R \in R$ does not obey
Normal Order (for set membership rather
than truth). This does not mean that no such set as $R$ exists; on the
contrary, in Gilmore's set theory it is provable that $R$ exists. 
Kripke provided a similar treatment for truth. Later
Feferman \cite{feferman:toward} and Perlis \cite{perlis:languages1,perlis:languages2} also working independently,
unified these two 
treatments.  The following schemata capture much of this approach:
$$T(\alpha) \iff \alpha *$$
where $\alpha *$ has no effect on $\alpha$ unless there is an embedded
and negated $T$ inside, and then:
$$(\neg T \beta)* \iff T \neg \beta$$
Finally we also require
$$\alpha * \implies \alpha$$
\noindent
That is, the assertion of the truth of the negated truth of an embedded
statement, amounts to the assertion of the truth of the
negation of the embedded statement. Put differently, $T$ pushes
negations through embedded $T$'s.  This seemingly convoluted affair,
in ``everyday'' cases of $\beta$, reduces to a triviality:

$$T(\neg T ~0=1) \iff T ~0 \not = 1 \iff 0 \not = 1$$

Thus---reading from right to left above---the bare fact of $ 0 \not =1$
then gives rise, in normal order, to the assertion that it is {\em
true}, and so on.
But for certain other cases things are more interesting. For instance,
here is what happens in the case of the $Liar$.

$$L \iff \neg T(L)$$
$$T(L) \implies T \neg T L \implies T \neg L \implies \neg L$$
yet also $T(L) \implies L$; so T(L) is contradictory and thus $\neg
T(L)$ is proven.  Also, $T \neg L \implies \neg L \implies T L \implies L$
so $T \neg L$ is contradictory and thus $\neg T \neg L$ is also
proven. Thus neither $L$ nor $\neg L$ is true, even though $L$ is
provable (it is equivalent to $\neg T L$ which we just proved). Thus a
provable wff need not be true (in the sense of the $T$-predicate).
Curious, but not contradictory.  The attempt to unpack $L$ into a bare
fact that stands independently prior to assessing $T(L)$ leads
nowhere, or rather, leads in a circle in such a way that we can
actually prove it cannot get a positive truth-judgment. 

Of course we would expect the above (from the Normal Order Principle),
since neither $L$ nor $\neg L$ separates enough to allow its truth clearly
to obtain or clearly to fail.  And the underlying formal treatment, following
Gilmore, is provably consistent, so it will not happen later on that
someone will discover a new conundrum (formalizable in the
Gilmore-Kripke setting) that leads to an outright inconsistency.
Thus one can have the $Liar$ and pacify its tendency toward
contradiction too.  The above two schemata for the Gilmore-Kripke
approach (GK for short) form a modification of Schema $T$, one that appears to
capture a sense of $T$ that comes close to the intuition behind the
Normal Order Principle.

\section{Flaws}

The $TruthTeller$  ought similarly to allow a proof that it does not
come out true, on the intuition underlying GK, that $T(x)$ signifies
that $x$ is judged true {\em after} $x$ is ascertained. Since one cannot
ascertain $TruthTeller$ before it is judged true (these are one and
the same thing) then the normal-order test fails, and so $\neg
T(TruthTeller)$ ought to result (and with it, $\neg TruthTeller$). But
formal means to show this failure of normal-order-ascertainment for
this sentence, are not present in existing treatments along the lines
of $GK$. 

Recall that the Tarski Hierarchy provides no means to express
$$This~ sentence~ has~ five~ words.$$
It is not that there is a difficulty in counting five words, nor of
recognizing (parsing) 
a sentence. ``Five'' and ``sentence'' can be given adequate formal
``meaning'' by appropriate axioms allowing derivation of the desired
results.  The difficulty is in associating {\em that very sentence
itself} with
the phrase ``This sentence''.  We can finesse it {\em a la} G\"odel,
using a special name or number $'S'$ for a particular sentence $S$.
That is, we can form a predicate $Five$ 
such that $Five(x)$ means $x$ is a sentence with five words, and via the
Diagonal Lemma there will be a sentence $S$ such that 
$\vdash S \iff Five(S)$. But the embedded $S$ (in suppressed quotes)
is a problem; it inhabits a level that must come
before the level of the main sentence, and yet at the same time it
occurs at that later level as well, a clear violation of the Tarski
Hierarchy.

Again, the underlying intuition behind GK presumably should provide a
solution to this problem by allowing a sentence $S$ such as above---in
the version in which $\vdash S \iff Five(S)$---to exist as an object
before its meaning (or truth, or fiveness) is determined. Then $S$ can
be counted to see whether $Five(S)$ holds. This even though normal
order has not entirely been respected here: we look to see, then we
should record the result, but the result is already recorded before we
look.  Still, the {\em meaning} of the $Five$ sentence can be
assessed, and its truth judged, after the sentence has been
recorded. The attempt to assess and judge does not circle back to that
very attempt, dooming it to failure; rather it leads to a counting and
a definite unique conclusion, even if that conclusion happens to be
already written down and indeed happens to be the very object being
counted. Yet no current $GK$-style formalization appears to have the
requisite machinery to capture this phenomenon.

The situation so far is this: GK allows standard examples of seeming
self-reference to be ``pseudo-''expressed, without actual
inconsistency, and in rough accordance with an intuitively sensible
principle. But there are flaws. The first is simply that certain
examples (such as the $TruthTeller$ and the $Five$ sentences) fit the
intuition behind the principle but the formalization does not fully
oblige, as we have seen. 

The second flaw is that the very
pseudo-expressings that allow GK to (appear to) treat self-reference
without banishing it, leave out the ingredient of (genuine)
reference. Finding a sentence $S$ that is provably equivalent to, say,
$P(S)$, does not in itself indicate that $S$ refers to anything at
all, let alone to itself. The deictic ``this'' of natural language (as
in ``This sentence is false.'') has been by-passed altogether in
formal treatments. Indeed, reference (or semantics) of any kind is
traditionally placed {\em outside} a formal language, as a function
defined on expressions in the language, mapping to an external
domain. The language in question does not typically have an expression
that stands in for this function; and even if it did, what would
determine that standing-in relation? It is as if meaning, or truth, is
always one step removed, leaning on some agreement lying outside
whatever language is used.  The GK approach does have a
truth-predicate, but it relies on an external naming convention that
leaves meaning, as a link between expressions $E$ and referents $O$,
unexpressed.

Regarding the first flaw, perhaps a remedy can be given in an
extended formalization of $GK$. Perhaps introduction of a $Means$
predicate along with some representation of temporal ordering can
provide a more explicit formal representation of (a version of) the
Normal Order Principle, allowing a treatment of the $TruthTeller$, as
well as the $Five$ sentence, to follow intuition.

In regard to the second flaw, it is only by means of an agreement
among whichever 
logicians happen to be participating in the discussion, that $'S'$
refers to anything at all, let alone to the sentence $S$, given, say, that
$S \iff T(S)$. For the $\iff$ relation
is not the same as a referring relation (to self or anything else).
After all, $0=1 \iff Snow~is~~composed~of~carbon$, but $0=1$ does not thereby
refer to anything, least of all to snow and carbon. Perhaps, again, some of
what is missing can be restored by introduction of a $Means$
predicate, as in $Means(E,...O...)$ to indicate (among other things)
that expression $E$ refers 
to object $O$. (But even this would beg the question as to who or what
takes $Means$ to so indicate---indicating already being a kind of
referring or meaning).
 
It would be of interest to formalize such a notion of meaning, as an
underpinning on which to analyze truth as a derivative notion.  For
example, one might attempt to characterize $T(x)$ in terms of
the meaning $m = (m_T,m_F)$ of $x$ where $m_T$ and $m_F$ are sets of
possible worlds (where $x$ is ``true'' or ``false'', resp.), $W$ is
the real world, and $W \in m_T$. Then $x$ 
might have a meaning even if it is not separate enough from that meaning to
be judged true or false (that is, if $W$ is in neither $m_T$ nor $m_F$).
Thus one would anticipate that for $tt=TruthTeller$, $W$ would
come out in the middle ($tt$ being neither true nor false in {\em any}
world)---as it already does in standard $GK$---and moreover that this
fact would be noted as $\neg T(tt)$ and $\neg F(tt)$, much like the $Liar$.
 
But meaning is a bit more tricky than this. For instance, $Blue(a)
\wedge \neg Blue(a)$ certainly seems meaningful, indeed quite definitely
false.  And its meaning seems just as definitely different
from that of, say, ${1 \not =1}$, which is also false.  Thus neither
sentence has a model in the conventional sense, and so neither would
be true in any world, and both would be false in all.  Their
meanings (in terms of pairs $m_T$ and $m_F$) then would be identical. This
is a complex issue (see \cite{frege:sense,meinong:possibility}); we will here only hint at an
idea that might be worth further exploration: Perhaps a semantics can
be developed that avails itself of superposed worlds, to allow one to
conceptualize a meaning for expressions such as $B \wedge \neg B$, e.g., a pair
of worlds where $B$ holds in one, and $\neg B$ in the other. Such a pair
for $Blue(a) \wedge \neg Blue(a)$ would not generally be the same as the
corresponding pair for, say, $ 1\not =1$. But the pair for the
latter presumably would be the same as for $1=1 \wedge 1 \not =1$,
namely $m_T=$ all worlds, and $m_F=\emptyset$, which seems
satisfying. 


Thus there is hope that the first flaw can be repaired, and the second
at least partially addressed, by taking meaning more seriously as a
concept to be formalized (and semanticized).  But the deeper aspect of
the second flaw remains: any kind of map between symbols and referents
is arbitrary, leaning on a decision to use that map, and not some
other, but an agent who intends to use that map.  Only when this issue
is faced head on, do we encounter genuine cases of reference, and the
possibility of genuinely self-referring expressions.


\section{No representation without representers!}

A lesson we draw with respect to the second flaw above is this:
self-reference proper has largely been 
left untouched by the very large literature purportedly on that
subject. This is because {\em reference} has largely been left
untouched, or rather pushed to the sidelines, via G\"odel numbering or
a similar artificial mechanism that leans on external agreements to
bring reference in at all. Attention has focussed, rather, on formal
counterparts of self-reference that do, to be sure, carry with them a
substantial potential for contradictoriness, in close analogy to their
informal---but more genuinely self-referential---sources.  But while
this attention has produced much of great importance, it has left much
out as well.  First and foremost is this problem: can there be
representation (meaning, reference) without an agent who chooses to so
represent?  And secondarily, what is the relation between reference in
general, and self-reference?  Third, what can be said about
``genuine'' informal self-referential expressions, in light of answers
to the former questions?

We will not attempt here to give a detailed analysis supporting a view
with respect to the first two questions. We will however state a position
(referring the reader to
\cite{perlis:putting1,perlis:putting2,perlis97:consciousness,perlis00:what}
for such an analysis):
Reference of $E$ to $O$ depends inherently on a referring agent $A$ to make the
connection between $E$ and $O$.  That is, the referring of $E$ to $O$ is
performed {\em by} $A$ rather than by the expression, $E$.  Moreover,
to perform a referring act, the agent $A$ must {\em take itself} to be
so referring, as part of that same act. Thus a referring act is
self-referring, and so self-reference is a necessary component of
reference of any kind.  This position is, admittedly, not coin of the
realm. But I believe it to be the only way to address head-on the
underlying issues. Some further motivation will be found by a brief
return to some earlier examples:

\begin{enumerate}
\item[N]
This sentence that I am now writing/uttering/expressing, beginning
with the word $This$, is false.

\item[W]
The only sentence written on the whiteboard at 4:19pm on May 12, 2003, in
room 3259 of the A. V. Williams Building at the University of
Maryland, is false.
\end{enumerate}

The seeming contradictoriness of $N$ (for $Now$)---and also of $W$ (for
Whiteboard) if that 
happened to be the only sentence on that whiteboard at that time---is
not the issue here.  The issue rather is what makes these sentences
self-refer.  In the case of (N) it appears straightforwardly to be the
presence of a self-referring agent ``I''. In the case of (W) there is
no immediately apparent agent, but without an agent (or even a whole
linguistic community of agents) {\em somewhere} in the broader
setting, these marks on a whiteboard have no meaning at all (after
all, they are in {\em English}). And that agent (or agents) will have to
self-refer in order to make sense of either (N) or (W), e.g., to
identify the University of Maryland as located in real physical space
situated somehow with respect to their own locations.


\section{A highly non-normal order}

We are now moving from the speculative to the ``more speculative'',
away from formal logic and toward philosophy.

A genuinely self-referring expression (or self-referring act) would
appear to severely violate normal order: it would refer to itself at the very
same time as it is expressed (or enacted).  That is, its very expression
(or enactment) would amount to its self-reference, making a temporal
sequencing seemingly out of the question.  Here are two more examples:

\centerline{\em I am now, with this very utterance, speaking English.}
\centerline{\em This is a hat.}

The latter does not seem to fit, unless we regard it as a gloss for
``This object I am calling your 
attention to now, with this very utterance, is a hat''.  Such a view
is---or is close to---one taken by Grice \cite{grice:meaning}, to the effect
that all utterances surreptitiously self-refer. Related observations have
been made by Millikan \cite{millikan:pushmi} and Perry
\cite{perry:problem}. On the 
basis of such considerations, it is argued in
\cite{perlis:putting1,perlis:putting2,perlis97:consciousness} that 
{\em all} intentional utterances are self-referential via their
self-referring utterer. That is, all cases of genuine referring are
cases of an agent who intentionally uses an expression to refer, and
who in so doing performs an act of self-referring akin to the hat
example above: ``the idea that I am intending, in this very act of
expression, is such and such.'' 

The above brief description of this claim does not do justice to the
underlying idea, and the reader is referred to just-cited papers for
additional argument. There it is further hypothesized that
thought in general, and not just communicative utterances, carry (and
do so more deeply) the underlying weight of the self-reference. 
To think a thought {\em is} to think its meaning, i.e., thoughts
carry their own meanings (whence the violation of normal order).  One
can then formulate various simple examples of self-referential {\em
thoughts} vaguely analogous to the $Liar$ and the $TruthTeller$, e.g.:

$$This~ is~ a~ thought.$$
$$This~ is~ not~ a~ thought.$$
$$I~ am~ (right~ now)~ thinking.I$$
$$I~ am~ not~ (right~ now)~ thinking.$$

Presumably---if taken as expressions of actual ongoing thoughts in
an agent---the first and third of these are necessarily true, and the
other two necessarily false.  
We can speculate further: as long as we are aware of anything at all,
to that extent we have a thought (to wit: there is an X), and then
there is an implicit ``That thing (that I am hereby noting before me) is an
X'' and so on. A self-referring self seems to underlie all thought or
awareness.  Thus a bare-bones agent self-reference may be the most basic
kind of reference.  What is bare-bones self-reference like? Imagine
yourself stripped little by little of this sensation, that thought,
until all that is left is your own grasp of being awake but not aware
of anything else.  This may be what is suggested in Piet Hein's poem
``EVENING AND MORNING SONG---About falling asleep and waking up''
\cite{hein:grooks}, with a dimming of awareness upon falling asleep;
or with the first glimmer of awareness upon waking up, in which one has
not yet recalled one is the person of yesterday with plans for
the day ahead, not yet at first identified with a name or a history or
a persona beyond the primitive self comprised only in the
self-referential and self-creating thought
that {\em this self-awareness is}:

\footnote{
My thanks to Torkil Heiede for bringing this poem to my attention.
}


\begin{quote}
The world disappears,\\
a loop running smaller, until\\
the thread is drawn out,\\
and the space it encloses is nil.\\ \\
\noindent
Newborn of nothing,\\
reluctantly starting to be,\\
fumbling awareness awakens\\
and finds that it's me.
\end{quote}

Such ideas, spun out far enough, have led to the hypothesis
\cite{perlis97:consciousness} that agent self-reference not only
underlies all reference, but indeed is tantamount to conscious
awareness.  Let us step back a little from such tenuous speculations,
to a more engineering perspective, pursuing an idea of John Perry 
\cite{perry:problem}.  There Perry describes pushing his shopping
cart along in an attempt to find the shopper whose cart is leaving a
trail of sugar on the floor, only later to realize  {\em he} is that
shopper. Setting aside various philosophical issues here, we end our
discussion by posing related questions about robot design.
Consider a robot that can decide {\em it} is the
robot who is, say, leaking oil, upon hearing that robot \#17 is leaking
oil. What is it for robot \#17 to know that it, itself, is that robot?
How does this affect its behavior?  Presumably it is quite important
to have such a capability, e.g., for survival.  See
\cite{perlis:intentionality,anderson/perlis:roots} for more elaborate
discussions of this idea.

In conclusion, the topic of self-reference appears to span a vast
intellectual territory, from formal logic to natural language, to
philosopy of mind, to artificial intelligence and robotics.  And very many
open questions remain.

%\end{verbatim}

\bibliographystyle{plain}
\bibliography{/fs/disco/group/bibfiles/ALL,/fs/disco/group/bibfiles/perlis1}



\end{document}
