
In this paper we hope to have supported the view that (explicit,
time-senstive, and flexible)
reasoning can be advantageous wherever there is uncertainty.  The
underlying architecture can be thought of as (one or more) procedural
algorithms with a suitably attached reasoning component
so the (system of interacting) algorithms becomes capable
of self-monitoring in real time. This affords certain protections
such as recognizing and repairing errors, informed guiding of
strategies, and incorporation of new concepts and terminologies.

In our view, this should apply almost across the board, not only to AI
programs, or to automated theorem-provers, but also, for example, to
operating systems.  Imagine an OS that had an attached active logic.
Such an OS-Alma pair not only could gather statistics on its ongoing
behavior (this is not new) but could infer and initiate real-time
alterations to its behavior as indicated by the circumstances. We
envision such an application as largely default-powered, along the
lines of error-diagnosis tools such as \cite{reiter-dekleer}. But an
active logic version has the additional virtues of accepting
dynamically changing observations, and of having a genuine real-time
default capability via its introspective lookup interpretation of
ignorance.


