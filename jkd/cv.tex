
    %%%%% MAKE A COPY OF THIS AS A WWW/USERS/PERLIS FILE %%%%%


\magnification=\magstep1
% fonts
%\font\rm=cmr10 scaled\magstep1
%\font\sl=cmsl10 scaled\magstep1
%\font\bf=cmbx10 scaled\magstep1
\font\ss=cmss10
%generate today's date
%
\def\today{\ifcase\month\or
 January\or February\or March\or April\or May\or June\or
 July\or August\or September\or October\or November\or December\fi
 \space\number\day, \number\year}
% the opposite of \nopagenumbers
\def\pagenumbers{\footline={\hss\sl --\folio -- \hss}}
%
% prevent bad page breaks
\raggedbottom\leftskip=.3 true in\rightskip=.3 true in\raggedright
\def\backup{\hskip -.5 true in}
%
% registers used and associated macros
%
\newcount\papno\papno=0 %set up paper number counter
 \def\paper{\advance\papno by 1\smallbreak
  \item{\number\papno .\ }}
\newcount\subsecnumber\subsecnumber=0 %set up subsection numbering
 \def\subsec#1\par{\medbreak\advance\subsecnumber by 1
  \noindent\backup{\ss #1}\par\nobreak\papno=0}
\newcount\subsubsecnumber\subsubsecnumber=0
 \def\subsubsec#1\par{\medbreak\advance\subsubsecnumber by 1
  \noindent\backup{\sl #1}\par\nobreak\papno=0}

\newcount\secnumber\secnumber=0 %set up section number counter
 \def\sec#1\par{\bigskip\filbreak\advance\secnumber by 1
  \subsecnumber=0\papno=0\subsubsecnumber=0
   \noindent\backup{\bf\uppercase\expandafter{\romannumeral\secnumber}.\ #1}
    \par\medbreak\nobreak}

% backpointers to papers from conference and tech reports sections
%name results for later reference
\global\def\BNAME#1
 {\expandafter\expandafter\edef\csname#1\endcsname{\number\papno}}
\def\BCITE#1{(subsumed in book chapter research publication
	\expandafter\csname#1\endcsname \ above)}
\global\def\JNAME#1
 {\expandafter\expandafter\edef\csname#1\endcsname{\number\papno}}
\def\JCITE#1{(subsumed in journal publication
	\expandafter\csname#1\endcsname \ above)}
%
% indented paragraph with left justified hanging tag (similar to .ip)
%   necessary since \hang will right justify the tag
%
% parameter 1 is the tag, either a single token or a group
% parameter 2 is the <dimen> of the offset
%
\def\inpar#1 #2 {\par\smallskip\hangindent=#2\noindent\rlap{#1}\hskip#2}
%
% expar (left extended paragraph)
%   set up a  paragraph such that the first line is not indented
%   and all other are
%
\def\expar{\smallskip\par\hang\noindent}
% the 'ole bottle of white-out
\overfullrule=0pt
%
%  Beginning of text
%
\rm\pagenumbers
\centerline{{\sl CURRICULUM VITAE}}
\centerline{\today}
\bigskip
\sec Personal Information

\bigskip
\halign{#\hfil&\qquad #\hfil\cr
  Donald~R.~Perlis\cr
  Professor\cr
  Computer Science Department and\cr
  Institute for Advanced Computer Studies\cr
  University of Maryland&~~~~~~~~~~~~~~~~~6907 Wells Parkway\cr
  College Park MD 20742&~~~~~~~~~~~~~~~~~University Park MD 20782\cr
  Office phone: (301) 405-2685&~~~~~~~~~~~~~~~~~Home phone: (301) 699-6207\cr
  Email: perlis@cs.umd.edu\cr
  Fax: (301) 405-6707\cr
  Url: www.cs.umd.edu/users/perlis\cr}
  

\bigskip
\subsec Education
\halign{#\hfil&\quad #\hfil&\hfil\quad #\quad\hfil&#\hfil\cr
\cr
  B.S.&Purdue University&1966&Mathematics\cr
\cr
  Ph.D.&New York University&1972&Mathematics (Advisor: Martin Davis)\cr
\cr
  Ph.D.&University of Rochester&1981&Computer Science (Advisor: James Allen)\cr}

\bigskip
\subsec University Experience
\medskip
\inpar {1997-} {.75in}
Professor, Computer Science Department, Univ.~of Maryland
\inpar {1988-1997} {.75in}
Associate Professor, Computer Science Department, Univ.~of Maryland
\inpar {1991-92} {.75in}
Visiting Associate Professor, Math.~Sciences Dept., Univ.~of Akron
\inpar {1988-89} {.75in}
Visiting Associate Professor, Computer Science Dept., Univ.~of Rochester
\inpar {1982-1988} {.75in}
Assistant Professor, Computer Science Department, Univ.~of Maryland
\inpar {1981-1982} {.75in}
Visiting Assistant Professor, Computer Science Dept., Univ.~of Delaware
\inpar {1980-1981} {.75in}
Research Assistant, Computer Science Department, Univ.~of Rochester
\inpar {1978-1980} {.75in}
Teaching Assistant, Computer Science Department, Univ.~of Rochester
\inpar {1977-1978} {.75in}
Assistant Professor with tenure, Mathematics Dept., U.~of Puerto Rico
\inpar {1972-1977} {.75in}
Assistant Professor, Mathematics Department, Univ.~of Puerto Rico

\sec Research, Scholarly and Creative Activities

\subsec Books Authored

\paper
{\sl Elementos  del  c\'alculo integral y diferencial},
(with R.~Wilson), a calculus textbook in Spanish,
Compa\~nia Editorial Continental, Mexico, 1979. 191 pages.
ISBN-968-26-0075-8.

%\subsec Books and Monographs in Preparation

%\paper
%Logic for a Lifetime: the hungry robot, and other roles of formalism
%in artificial intelligence

%\paper
%Integrative Mathematics


\subsec Chapters in Refereed Books

\paper
On the consistency of commonsense reasoning. 
{\sl Readings in Non-Monotonic Reasoning}, edited
by M.~Ginsberg. Morgan Kaufmann, 1987, pp.~56--66
(reprinted from journal paper 9 below).

\paper\BNAME{CST}
Commonsense set theory.
{\sl Meta-level Architectures and Reflection}, edited
by P.~Maes and D.~Nardi. North-Holland, 1988, pp.~87--98.

\paper\BNAME{META}
An overview of meta in logic.
Invited survey article, in
{\sl Meta-level Architectures and Reflection}, edited
by P.~Maes and D.~Nardi. North-Holland, 1988, pp.~37--49.

\paper
Thing and thought.
Invited chapter, 
{\sl Knowledge Representation and Defeasible Reasoning},
edited by H.~Kyburg, R.~Loui, and G.~Carlson.
Kluwer, 1990, pp.~99--117.

\paper
Memory, reason, and time: the step-logic approach.
(With J.~Elgot-Drapkin and M.~Miller.)
Invited chapter, {\sl Philosophy and AI: Essays at the Interface},
edited by R.~Cummins and J.~Pollock, MIT Press, 1991.

\paper
Limited scope and circumscriptive reasoning.
(With D.~Etherington and S.~ Kraus.)
Invited chapter,
{\sl Advances in Human and Machine Cognition,
vol.~1: the Frame Problem in Artificial Intelligence},
K.~Ford and P.~Hayes (eds.), JAI Press, 1991
(reprinted from journal paper 18 below).

\paper
Intentionality and defaults.
Invited chapter,
{\sl Advances in Human and Machine Cognition,
vol.~1: the Frame Problem in Artificial Intelligence},
K.~Ford and P.~Hayes (eds.), JAI Press, 1991
(reprinted from journal paper 19 below).

\paper
Languages with self-reference I: Foundations. Invited
chapter,
{\sl Reflexivity: A Source-Book in Self-Reference},
S. J. Bartlett (ed.), North-Holland, 1992
(reprinted from journal paper 6 below).

\paper
Languages with self-reference II: Knowledge, belief,
and modality. Invited chapter,
{\sl Reflexivity: A Source-Book in Self-Reference},
S. J. Bartlett (ed.), North-Holland, 1992
(reprinted from journal paper 12 below).

\paper
Metalanguages, reflection principles and self-reference.
(With V.S. Subrahmanian.)
Invited chapter,
{\sl Handbook of Logic in Artificial Intelligence and Logic
Programming, vol.~2: Deduction Methodologies}, D. Gabbay, C.J. Hogger,
and J.A. Robinson (eds.), Oxford University Press, 1994.

\paper
Putting one's foot in one's head---Part II: How.
Invited chapter, 
{\sl From Thinking Machines to Virtual Persons:
Essays on the Intentionality of Computers},
E.~Dietrich (ed.), Academic Press, 1994.

\paper
Explicitly biased generalization.
(With D.~Gordon.) Invited chapter,
{\sl Goal-Driven Learning}, A.~Ram and D.~Leake (eds.),
MIT Press, 1995
(reprinted from journal paper 16 below).

\paper
%What experts deny, novices must understand.
Toward Automated Expert Reasoning and Expert-Novice Communication.  (With
M. Miller.)  Invited chapter, {\sl Expertise in
Context: Human and Machine}, K. Ford, P. Feltovich, and R. Hoffman
(eds). MIT Press, 1997.

\paper
The role(s) of belief in AI.  Chapter 14 in
J. Minker (ed.) {\sl Logic-based AI}, Kluwer. 2000.

\paper
Theory and application of self-reference: logic and beyond.
To appear as chapter in book, publ Center for Study of Language and
Information (CSLI, Stanford).

%\subsubsec ~~Submitted for Publication

\subsec Articles in Refereed Encyclopedia

\paper
Circumscription.
{\sl The Encyclopedia of Artificial Intelligence}, edited
by S.~Shapiro. Wiley, 1987, pp.~100--103. Updated for 2nd edition, 1991.

\paper
Nonmonotonic reasoning.
{\sl The Encyclopedia of Artificial Intelligence}, edited
by S.~Shapiro. Wiley, 1987, pp.~849--853. Updated for 2nd edition,
1991.

\paper Symbol Systems. Michael L. Anderson and Don
Perlis.  {\sl Encyclopedia of Cognitive Science}, 2002.

\subsec Special Issue of Journal Edited

\paper
Context: Theory and Practice. A collection of eight refereed, previously
unpublished papers by various authors, edited and
with an introduction by D.~Perlis. Special issue of {\sl Fundamenta
Informaticae}, vol.~23, pp.~145--396, 1995.

\subsec Articles in Refereed Journals

\paper
An extension of Ackermann's set theory.
{\sl Journal of Symbolic Logic}, 
vol.~37, 1972, pp.~703--704.

\paper
Group algebras and model theory.
{\sl Illinois Journal of Mathematics}, 
vol.~20, 1976, pp.~298--305.

\paper
An application of compiler simulation at the source language level.
{\sl The Computer Journal}, 
vol.~19, 1976, p.~90.

\paper
Utility functions, public goods, and income redistribution.
(With A.~Mann.)
{\sl Public Finance Quarterly}, 
vol.~5, 1977, pp.~9--22.

\paper
A re-evaluation of story grammars.
(With A.~Frisch.)
{\sl Cognitive Science},
vol.~5, 1981, pp.~79--86.

\paper\JNAME{AIJ-I}
Languages with self-reference I: foundations.
{\sl Artificial Intelligence},
vol.~25, 1985, pp.~301--322.
(Reviewed in {\sl Computing Reviews}, May 1986, pp.~266--267.
Listed in D.~Bobrow's 1993 compilation
of the most frequently-cited papers in {\sl Artificial Intelligence}
from 1970 to 1991. Central result was later selected for use in KIF:
Knowledge Interchange Format, a document of the Interlingua Working
Group of the ARPA Knowledge Sharing Effort, printed as a report of
the Logic Group, CS Dept, Stanford University).

\paper\JNAME{JLP}
Computing protected circumscription.
(With J.~Minker.)
{\sl Journal of Logic Programming}, vol.~4, 1985, pp.~235--249.

\paper\JNAME{AIJ-M}
Completeness results for circumscription.
(With J.~Minker.)
{\sl Artificial Intelligence}, vol.~28, 1986, pp.~29--42.

\paper\JNAME{OCCR}
On the consistency of commonsense reasoning.
{\sl Computational Intelligence},
vol.~2, 1986, pp.~180--190.

\paper
Circumscribing with sets.
{\sl Artificial Intelligence},
vol.~31, 1987, pp.~201--211.

\paper
Proving self-utterances.
(With M.~Miller.)
{\sl Journal of Automated Reasoning}, 
vol.~3, 1987, pp.~329--338.

\paper\JNAME{AIJ-II}
Languages with self-reference II: knowledge, belief, and modality.
{\sl Artificial Intelligence},
vol.~34, 1988, pp.~179--212.

\paper
Autocircumscription.
{\sl Artificial Intelligence},
vol.~36, 1988, pp.~223--236.

\paper
Uniform accountability for multiple modes of reasoning.
(With L.~Kanal.)
{\sl International Journal of Approximate Reasoning},
special issue on Uncertainty in Artificial Intelligence. T.~Levitt (ed.),
vol.~2, 1988, pp.~233--246.

\paper
Truth and meaning.
{\sl Artificial Intelligence},
vol.~39, 1989, pp.~245--250.

\paper
Explicitly biased generalization.
(With D.~Gordon.)
{\sl Computational Intelligence},
vol.~5, 1989, pp.~67--81.

\paper
Reasoning situated in time I: basic concepts.
(With J.~Elgot-Drapkin.)
{\sl J. of Experimental and Theoretical Artificial Intelligence},
vol.~2, 1990, pp.~75--98.

\paper
Limited scope and circumscriptive reasoning.
(With D.~Etherington and S.~ Kraus.)
{\sl International J. of Expert Systems},
special issue on the Frame Problem, Part A.
K.~Ford and P.~Hayes (eds.),
vol.~3, 1990, pp.~207--217.

\paper
Intentionality and defaults.
{\sl International J. of Expert Systems},
special issue on the Frame Problem, Part B.
K.~Ford and P.~Hayes (eds.), 
vol.~3, 1990, pp.~345--354.

\paper
Stop the world---I want to think.
(With J.~Elgot-Drapkin and M.~Miller.)
Invited paper, {\sl International J. of Intelligent Systems},
special issue on Temporal Reasoning, K.~Ford and F.~Anger (eds.),
vol.~6, 1991, pp. 443--456.

\paper
Nonmonotonicity and the scope of reasoning.
(With D.~Etherington and S.~Kraus.)
{\sl Artificial Intelligence},
vol.~52, 1991, pp.~221--261.

\paper
Putting one's foot in one's head---Part I: Why.
{\sl No\^{u}s},
special issue on Artificial Intelligence and Cognitive Science,
W.~Rapaport (ed.), vol.~25, 1991, pp.~435--455.

\paper
Reasoning about ignorance: A note on the Bush--Gorbachev problem.
(With J.~Horty and S.~Kraus.)
{\sl Fundamenta Informaticae},
special issue on Logic for Artificial Intelligence,
Z.~Ras (ed.), vol.~15, 1991, pp.~325--332.

\paper
Logic and artificial intelligence: a new synthesis?
{\sl Fundamenta Informaticae},
special issue on Algebraic Methods in Logic and their Computer Science
Applications, C.~Rauszer (ed.), vol.~18, 1993, pp.~297--305.

\paper
Consciousness and complexity: the cognitive quest. Invited paper,
{\sl Annals of Mathematics and Artificial Intelligence}, special issue in
honor of Jack Minker, vol.~14, 1995, pp.~309--321.

\paper
Automated inference in active logics.
(With M. Miller.) Invited paper, special issue of {\sl Journal
of Applied Non-Classical Logics}, vol.~6, 1996, pp.~9--27.

\paper
How to (plan to) meet a deadline between {\sl now} and {\sl then}.
(With S.~Kraus, M.~Miller and M.~Nirkhe.) {\sl J.~of Logic and Computation},
1997, vol.~7.

\paper
Sources of, and exploiting, inconsistency: preliminary report.
Special issue of {\sl Journal of Applied Non-Classical Logics}, 1997.
(Invited reprint of conference paper 38 below.)

\paper
Interpreting presuppositions using active logic: from contexts to
utterances.
(With J.~Gurney and K.~Purang.) {\sl Computational Intelligence}.
1997, vol.~13, pp.~391--413.

\paper
Consciousness as self-function. Invited paper for special issue
of {\sl Journal of Consciousness Studies}. 1997, vol. 4, pp.~509-25.

\paper
Conversational Adequacy: Mistakes are the essence.
{\sl International Journal of Human Computer Studies}. 1998, vol. 48.
(With K. Purang and C. Andersen.)

\paper
Representations of dialogue state for domain and task independent
meta-dialogue. 
{\sl Electronic
Transactions in Artificial Intelligence}. (With C. Andersen, W. Chong,
D. Josyula, M. O'Donovan-Anderson, K. Purang, and D. Traum.)
1999, Vol. 3, Section D, pp 125--152.
http://www.ida.liu.se/ext/epa/ej/etai/1999/D/index.html

\paper
What does it take to refer?
{\sl J. of Consciousness Studies}. 2000, vol.~7, pp.~67--9.

\paper
A logic for characterizing multiple bounded agents.
(With S. Kraus and J. Grant.) {\sl Autonomous Agents and
Multi-Agent Systems}. 2000, vol.~3, pp.~351--387.

\paper
The roots of self-awareness.
(With Michael Anderson.)  To appear, {\sl Phenomenology and the Cognitive
Sciences}.

\paper
A logic-based model of intention formation and action for multi-agent
subcontracting.
(With John Grant and Sarit Kraus.) To appear, {\sl Artificial Intelligence}.

\paper
Logic, self-awareness and self-improvement: the metacognitive loop and
the problem of brittleness.
(With Michael Anderson.) To appear, {\sl Journal of Logic and Computation}.


%\subsubsec ~~Submitted for Publication

%\paper
%Defaults denied. (With M. Miller and K.~Purang.)  Submitted to {\sl Artificial
%Intelligence}.

%\subsubsec ~~In Preparation

%\paper
%Active logics: A Unified Formal Approach to Episodic Reasoning. (With J.~Elgot-Drapkin, S.~Kraus, M.~Miller, and M.~Nirkhe.)
%
%\paper
%Logic for a lifetime.

%\paper
%Self-computing: a general theory of intelligence and mind. To
%be submitted, {\sl Behavioral and Brain Sciences}.

%\paper
%Rapid semantic shift.
%(With M. Miller.) To be submitted, {\sl Cognitive Science}.

%\paper
%Putnam's Theorem and the intentionality machine.

%\paper
%Cognitive adequacy.

\subsec Presentations at Refereed Conferences

\paper
Applications  of  protected  circumscription.
(With J.~Minker.)
{\sl Proceedings, Seventh Conference on Automated Deduction, May 1984},
Lecture Notes in Computer Science, Springer-Verlag, 1984.

\paper
Protected circumscription.
(With  J.~Minker.)
{\sl Proceedings, Workshop on Non-Monotonic Reasoning},
New Paltz, October 1984.

\paper
Non-monotonicity and real-time reasoning.
{\sl Proceedings, Workshop on Non-Monotonic Reasoning}, 
New Paltz, October 1984.

\paper
Circumscription: completeness, computation, and commonsense.
(With J.~Minker.)
{\sl Workshop on Logic and Computer Science},

Lexington, Kentucky, June 1985.

\paper
What is and what isn't.
Invited paper,
{\sl 12th Annual Meeting, Society for Philosophy and Psychology},
Johns Hopkins, June, 1986.

\paper
Step-logics: an alternative approach to limited reasoning.
(With J.~Drapkin.)
{\sl Proceedings, 7th European Conference on Artificial Intelligence},
Brighton, England, July 1986.

\paper
Self-reference, knowledge, belief, and modality.
{\sl Proceedings, AAAI-86},
Philadelphia; nominated for Publisher's Prize
\JCITE{AIJ-II}.

\paper
A parallel self-modifying default reasoning system.
(With J.~Minker and K.~Subramanian.)
{\sl Proceedings, AAAI-86},
Philadelphia.

\paper
A preliminary excursion into step-logics.
(With J.~Drapkin.)
{\sl Proceedings, Intl Symp on Methodologies for Intelligent Systems},
October 1986, Knoxville, Tennessee.

\paper
Commonsense set theory.
{\sl Proceedings, Workshop on Meta-level Architectures and Reflection},
October 1986, Sardinia
\BCITE{CST}.

\paper
An overview of meta in logic.
{\sl Proceedings, Workshop on Meta-level Architectures and Reflection},
October 1986, Sardinia
\BCITE{META}.

\paper
The two frame problems.
(With J.~Elgot-Drapkin and M.~Miller.)
{\sl Proceedings, Workshop on the Frame Problem},
April 1987, Lawrence, Kansas.

\paper
Life on a desert island.
(With J.~Elgot-Drapkin and M.~Miller.)
{\sl Proceedings, Workshop on the Frame Problem},
April 1987, Lawrence, Kansas.

\paper
How can a program mean?
{\sl Proceedings, International Joint Conference on Artificial Intelligence,}
August, 1987, Milan, Italy.

\paper
Proving facts about `I'.
(With M.~Miller.)
{\sl Proceedings, International Joint Conference on Artificial Intelligence,}
August, 1987, Milan, Italy.

\paper
Circumscription as introspection.
{\sl Proceedings, Second Intl Symp on Methodologies for Intelligent Systems}.
October, 1987, Charlotte, North Carolina.

\paper
Thing and thought.
Invited paper,
Annual conference of the
{\sl International Society for Exact Philosophy},
on Natural Philosophy and Artificial Intelligence: Logic and Language,
June, 1988, Rochester, NY.

\paper
Limited scope and circumscriptive reasoning.
(With D.~Etherington and S.~Kraus.)
{\sl First International Workshop on Human and Machine
Cognition},
May, 1989, Pensacola, Florida.

\paper
Intentionality and defaults.
Invited paper,
{\sl First International Workshop on Human and Machine
Cognition},
May, 1989, Pensacola, Florida.

\paper
Assessing others' knowledge and ignorance.
(With S.~Kraus.)
{\sl Proceedings, Fourth
Intl Symp on Methodologies for Intelligent Systems}.

\paper
Nonmonotonicity and the scope of reasoning: preliminary report.
(With D.~Etherington and S.~Kraus.)
{\sl Proceedings, AAAI-90}, pp.~600--607.

\paper
Planning and acting in deadline situations.
(With M.~Nirkhe and S.~Kraus.)
{\sl AAAI-90 Workshop on Automated Planning for Complex Domains.}

\paper
Deadline-coupled real-time planning.
(With S.~Kraus and M.~Nirkhe.)
{\sl Proceedings, DARPA Workshop on Innovative Approaches to Planning,
Scheduling, and Control}, 1990.

\paper
Fully deadline-coupled planning: one step at a time.
(With M.~Nirkhe and S.~Kraus.)
{\sl Proceedings, sixth
Intl Symp on Methodologies for Intelligent Systems},
Charlotte, NC, 1991.

\paper
Typicality constants and range defaults: the pros
and cons of a cognitive model of default reasoning.
(With M.~Miller.)
{\sl Proceedings, Sixth
Intl Symp on Methodologies for Intelligent Systems},
Charlotte, NC, 1991.

\paper
Logic and AI: a new synthesis? Invited
as a main speaker, Stefan Banach International Mathematical
Center, {\sl Semester on Algebraic Methods in Logic and Their
Computer Science Applications,} November 1991, Warsaw, Poland.

\paper
Memory, mind, and models of self. One-hour invited
lecture, {\sl Canadian Society for Computational Studies of
Intelligence}, presented at AI-92 conference, May 1992, Vancouver.

\paper
Situated reasoning within tight deadlines and realistic space and
computation bounds. 
(With M.~Nirkhe and S.~Kraus.)
{\sl Second Symposium on Logical Formalizations of Commonsense Reasoning},
1993. Also presented at the Workshop on Spatial and Temporal
Reasoning, IJCAI-93.

\paper
Reasoning about change in a changing world.
(With M.~Nirkhe and S.~Kraus.)
{\sl Proceedings of the Florida AI Research Symposium---FLAIRS-93},
Fort Lauderdale, 1993.

\paper
Presentations and this and that: logic in action.
(With M. Miller.) Cog-Sci-93. Boulder, Colorado, 747-752. Also
appeared as a AAAI Symposium workshop paper, Raleigh, 1993.

\paper
Vacuum logic.
(With J. Elgot-Drapkin, S. Kraus, M. Miller, M. Nirkhe.)
AAAI Fall Symposium on Instantiating Real-World Agents.
Raleigh, October 1993.

\paper
What experts deny, novices must understand.
(With M. Miller.) {\sl Third International Workshop on 
Human \& Machine Cognition}, 1993, Seaside, Florida.

\paper
Recognition of object functionality in goal-directed robotics.
(With E.~Rivlin and A.~Rosenfeld.)  Workshop on Reasoning About Function,
AAAI-93, Washington, DC.

\paper
An error-theory of consciousness. {\sl Toward a scientific basis for
consciousness}, conference held in Tucson, 1994.

\paper
Calibrating, counting, grounding, grouping.
(With J. Elgot-Drapkin, D. Gordon, S. Kraus, M. Miller, M.
Nirkhe.) AAAI Fall Symposium on Control of the Physical World 
by Intelligent Agents. New Orleans, November 1994.

\paper
Thinking takes time: A modal active-logic for reasoning {\sl in} time.
(With M.~Nirkhe and S.~Kraus.) BISFAI-95 (Bar Ilan
Symposium on Foundations of Artificial Intelligence).

\paper
Active logic and Heim's rules for updating discourse context.
(With J.~Gurney and K.~Purang.) IJCAI-95 Workshop on
context in natural language processing. Montreal, 1995.

\paper
Sources of, and exploiting, inconsistency: preliminary report.
Common Sense 96 (Third Symposium on Logical Formalizations of
Commonsense Reasoning) Stanford, January 6-8, 1996.

\paper
Active Logic Applied to Cancellation of Gricean Implicature.
(With J.~Gurney and K.~Purang.) AAAI Spring Symposium on
Computational Implicature. Stanford, March 25-27, 1996.

\paper
Conversational Adequacy: Mistakes are the Essence.  (With K.~Purang.)
AAAI Workshop on Detecting, Repairing, and Preventing Human-machine
Miscommunication, Portland, Oregon, August 1996.

\paper
The WHs of NCC.  Annual Meeting of the Association for the Scientific
Study of Consciousness. Bremen, Germany, June, 1998.

\paper
Consciousness, Self, and Meaning.
Annual Meeting of the Association for the Scientific
Study of Consciousness. London, Ontario, June, 1999.

\paper
Modeling Time and Meta-Reasoning in Dialogue Via Active Logic.
(With K. Purang, D. Purushothaman, C. Andersen, D. Traum.)
1999 AAAI Fall Symposium, Psychological Models of Communication in
Collaborative Systems (full paper).

\paper
Mixed Initiative Dialogue and Intelligence via Active Logic.
(With C. Andersen, D. Traum, K. Purang, D. Purushothaman.)
AAAI 99 Workshop on Mixed-Initiative Intelligence.

\paper
Practical Reasoning and Plan Execution with Active Logic.
(With K. Purang, D. Purushothaman,  D. Traum, C. Andersen.)
1999 IJCAI Workshop on Practical Reasoning and Rationality.

\paper
Seven Days in the Life of a Robotic Agent. {With Waiyian Chong, Michael
O'Donovan-Anderson, and Yoshi Okamoto.)
GSFC/JPL Workshop on Radical Agent Concepts. 2001. NASA Goddard Space
Flight Center, Greenbelt, MD, USA. 

\paper
Handling Uncertainty with Active Logic. (With Manjit Bhatia, Paul Chi,
Waiyian Chong, Darsana P. Josyula, Michael O'Donovan-Anderson, Yoshi
Okamoto, and K. Purang.) Proceedings, AAAI Fall Symposium on
Uncertainty in Computation. 2001.

\paper
An information integration environment based on the active logic 
framework. (With A A Barfourosh, H R Mothary Nezhad, M O'Donovan-
Anderson.)  MIS 2002.

\paper
A Logic-Based Model of Intentions for Multi-Agent Subcontracting.
AAAI-2002 (26\% acceptance rate this year.) (With John
Grant and Sarit Kraus.)

\paper
Time-Situated Agency: Active Logic and Intention Formation.
Cognitive Agents Workshop, 2002.  With M. Anderson,
D. Josyula, and Y. Okamoto.)

\paper
The Use-Mention Distinction and its importance to {H}{C}{I}. (With Michael
L. Anderson and Yoshi Okamoto.) Proceedings of
the Sixth Workshop on the Semantics and Pragmatics of
Dialog. 2002.

\paper
Towards domain-independent, task-oriented, conversational
adequacy. (With Darsana Josyula and Michael Anderson.) IJCAI-2003,
Intelligent Systems Demonstration, Acapulco Mexico.

\paper
Talking to computers. (With Darsana Josyula and Michael Anderson.)
IJCAI-2003, Proceedings of the Workshop on Mixed Initiative
Intelligent Systems Acapulco Mexico.

\paper
RGL Study In A Hybrid Real-Time System. (With K. Hennacy and N. Swamy.)
IASTED NCI, 2003, Cancun Mexico.

\paper
Active Logic for more effective human-computer interaction and other
commonsense applications. 
(With Michael L. Anderson, Darsana Josyula, Khemdut Purang.)
International Joint Conference on Automated Reasoning,
Workshop on Empirically Successful First Order Reasoning, Ireland, 2004. 


%\subsubsec ~~Submitted for Publication

%\paper
%Logic for a lifetime: preliminary report. Submitted to IJCAI-95.

%\subsec ~~In Preparation

\subsec Reports

\paper
Ackermann's set theory and related topics.
New York University (Ph.D. thesis), 1971.

\paper
Distance  spaces  and natural convexity.
(With D.~Hajek and R.~Wilson.)
Technical Report, National University of Mexico, 1978

\paper
A heuristic calculation of the utility of income.
(With L.~Delsanto.)
University of Puerto Rico, 1978.

\paper
Physical theory and the divisibility of space and time.
(With R.~Sarraga.)
University of Puerto Rico, 1978.

\paper
Truth and syntax.
Technical Report, Computer Science Department,
University of Rochester, 1979. (Revised as Truth, syntax, and reason,
June and August, 1980.)

\paper
Language, computation, and reality.
Technical Report, Computer Science
Department, University of Rochester (Ph.D. Thesis) 1981.

\paper
A Tarskian truth definition.
Manuscript, 1982.

\paper
Truth as normal-order principle.
Manuscript, 1982.

\paper
A magical inference number.
Manuscript, 1983.

\paper
True beliefs.
Computer Science Department Technical Report, University of Maryland, 1984.
\JCITE{AIJ-I}

\paper
Completeness results for circumscription.
(With J.~Minker.)
Computer Science Department Technical Report, University of Maryland, 1985.
\JCITE{AIJ-M}

\paper
On the consistency of commonsense reasoning.
Computer Science Department Technical Report,
University of Maryland, 1985.

\paper
Real-time default reasoning, relevance, and memory models.
(With J.~Drapkin and M.~Miller.)
Systems Research Center Technical Report,
University of Maryland, 1985.

\paper
A commentary on the literature of self-reference.
(With M.~Miller.)
Systems Research Center Technical Report,
University of Maryland, 1985.

\paper
Default handling: consistency before and after.
(With J.~Drapkin and M.~Miller.)
Systems Research Center Technical Report,
University of Maryland, 1985.

\paper
A memory model for real-time commonsense reasoning.
(With J.~Drapkin and M.~Miller.)
Systems Research Center Technical Report,
University of Maryland, 1986.

\paper
Analytic completeness in $SL_0$.
(With J.~Drapkin.)
Comp. Sci. Technical Report,
University of Maryland, 1986.

\paper
Languages with self-reference II: knowledge, belief, and modality.
Institute for Advanced Computer Studies TR-87-13,
Computer Science Dept CS-TR-1815,
University of Maryland, 1987
\JCITE{AIJ-II}.

\paper
Autocircumscription.
Institute for Advanced Computer Studies TR-88-28,
Computer Science Dept CS-TR-2014,
University of Maryland, 1988.

\paper
Reasoning situated in time I: basic concepts.
(With J.~Elgot-Drapkin.)
Institute for Advanced Computer Studies TR-88-29,
Computer Science Dept CS-TR-2016,
University of Maryland, 1988.

\paper
Names and non-monotonicity.
(With S.~Kraus.)
Institute for Advanced Computer Studies TR-88-84,
Computer Science Dept CS-TR-2140,
University of Maryland, 1988.

\paper
Some brief essays on mind.
Technical Report 302,
Computer Science Department,
University of Rochester, 1989

\paper
Stop the world---I want to think.
(With J.~Elgot-Drapkin and M.~Miller.)
Institute for Advanced Computer Studies TR-90-26,
Computer Science Dept CS-TR-2415,
University of Maryland, 1990.

\paper
Nonmonotonicity and the scope of reasoning.
(With D.~Etherington and S.~Kraus.)
Institute for Advanced Computer Studies TR-90-56,
Computer Science Dept CS-TR-2457,
University of Maryland, 1990.

\paper
Putting one's foot in one's head---Parts I and II.
Institute for Advanced Computer Studies TR-91-58,
Computer Science Dept CS-TR-2659,
University of Maryland, 1991.

\paper
Consciousness and complexity: the cognitive quest.
CS-TR-3232, UMIACS-TR-94-25
March 1994.

\paper
Presentations and this and that: logic in action.
(With M. Miller.)
CS-TR-3244, UMIACS-TR-94-36
March 1994.

\paper
Thinking takes time: A modal active-logic for reasoning {\sl in} time.
(With M.~Nirkhe and S.~Kraus.) 
CS-TR-3249, UMIACS-TR-94-39
March 1994.

\paper
Logic for a lifetime.
CS-TR-3278, UMIACS-TR-94-62
May 1994.

\paper
Calibrating, counting, grounding, grouping.
(with J. Elgot-Drapkin, D. Gordon, S. Kraus, M. Miller, and M.
Nirkhe.)
CS-TR-3279, UMIACS-TR-94-63
May 1994.

\paper
What experts deny, novices must understand.
(With M. Miller.)
CS-TR-3280, UMIACS-TR-94-64
May 1994.

\paper
An error-theory of consciousness.
CS-TR-3324, UMIACS-TR-94-91
July 1994.

\paper
Conversational Adequacy: Mistakes are the Essence.
(With K. Purang.)
CS-TR-3654, UMIACS-TR-96-41
June 1996.

\paper
Active Logic Applied to Cancellation of Gricean Implicature.
(With J. Gurney and K. Purang.)
CS-TR-3655, UMIACS-TR-96-42
June 1996.

\paper
Active Logic and Heim's Rules for Updating Discourse Context.
(With J. Gurney and K. Purang.)
CS-TR-3656, UMIACS-TR-96-43
June 1996.

\paper
Active logics: A unified formal approach to episodic
reasoning. (With Jennifer Elgot-Drapkin, Sarit Kraus, Michael Miller,
Madhura Nirkhe.) CS-TR-4072. October 1999.

\subsec Book Reviews, Other Articles, and Notes

\paper
Whose category error?  (peer commentary)
{\sl Behavioral and Brain Sciences},
vol.~6, 4, 1983, pp.~606--607.

\paper
Belief  level  way  stations.  (peer commentary)
{\sl Behavioral and Brain Sciences},
vol.~7, 4, 1984, pp.~639--640.

\paper
Intentionality as internality. (peer commentary)
(With R.~Hall.)
{\sl Behavioral and Brain Sciences},
vol.~9, 1, 1986, pp.~151--152.

\paper
Discussion of Cheeseman's ``An inquiry into computer
understanding.'' (With L.~Kanal.)
{\sl Computational Intelligence},
vol.~4, 1, 1988, pp.~87--89.

\paper
The emperor's old hat. (peer commentary) 
{\sl Behavioral and Brain Sciences}, vol.~13, 4, 1990, pp.~680--681


\subsec Talks, Abstracts, and other Professional Papers Presented

\subsubsec ~~Invited lectures at refereed conferences

\expar
Memory, mind, and models of self. Invited one-hour lecture, Canadian
AI Conference, Vancouver, May 1992.

\expar
Putnam's Theorem and the intentionality machine. Invited one-hour
lecture, Society for Machines and Mentality, Washington D.C., December
1992. 


\subsubsec ~~~Invited talks, general

%\expar
%A survey of mathematical logic. Univ.~of Puerto Rico, 1973

%\expar
%Transfinite arithmetic. Univ.~of Puerto Rico, 1974

%\expar
%Formal theories. CURCAM, Sto. Domingo, 1977

%\expar
%Self-reference and robotics. Naval Research Laboratory, 1983

%\expar
%Panel on Artificial Intelligence, ACM Chapter, Univ.~of Md., 1983

%\expar
%Overview of artificial intelligence. Education seminar, Univ.~of Md., 1983

\expar
Semantics of circumscription. Stanford University, May 1984

\expar
A magical inference number. National Bureau of Standards, June 1984

\expar
A direct manipulation language. IBM Thomas Watson Laboratory, July 1984

\expar
Topics in circumscription. University of Sussex, Jan. 1985

\expar
Topics in circumscription. University of Edinburgh, Jan. 1985

\expar
Logic of belief. Louisiana State University, Aug. 1985

\expar
Logic of belief. Univ.~of Toronto, Dec. 1985

\expar
Logic of belief. University of Rochester, Dec. 1985

\expar
Panel on Machine Learning, ACM Chapter, Univ.~of Md., Feb. 1986

\expar
A theory of absolute universes. Univ.~of Tennessee, May, 1986

\expar
An overview of meta in logic. Sardinia, Oct. 1986

\expar
Knowledge representation. Smithsonian Institution, July 1987

\expar
Self reference and artificial intelligence. Naval Research Laboratory, May 
1988

\expar
Thing and thought. SUNY at Buffalo, Nov. 1988

\expar
Self reference and artificial intelligence. Univ.~of Akron, Nov. 1988

\expar
Meaners. ATT Bell Labs, Mar. 1989

\expar
Russell's paradox is alive and well.
Univ.~of Akron, Mar. 1989

\expar
Intentionality. Univ.~of Pittsburgh, Mar. 1989

\expar
Issues in circumscription. Univ.~of Pittsburgh, Mar. 1989

\expar
The four references. SUNY at Buffalo, Mar. 1989

\expar
Scope and nonmonotonicity. Stanford Univ., Aug. 1990

\expar
Planning and perception. SRI International, Aug 1990

\expar
Intentionality. Xerox PARC, Aug 1990

\expar
Logic and artificial intelligence: a new synthesis? Univ.~of Akron,
Nov 1991

\expar
Logic and artificial intelligence: a new synthesis? Banach Center,
Warsaw, Poland, Nov 1991

\expar
Memory, mind, and models of self. Invited one-hour lecture, Canadian
AI Conference, Vancouver, May 1992.

\expar
Putnam's Theorem and the intentionality machine. Invited one-hour
lecture, Society for Machines and Mentality, Washington D.C., December
1992. 

\expar
Logic, robots, and minds.
Computer Engineering Seminar Series,
Purdue University, March 1994.

\expar
Logic for a lifetime.
Computer Science Colloquium Series,
Harvard University, March 1995.

\expar
Testable aspects of a self-theory of mind and brain.
Cognitive Neuroscience Section,
National Institutes of Health, November 1996.

\expar
Logic for a Lifetime.
Technische Universitat, Darstadt, Germany, June 1998.

\expar
Logic for a Lifetime.
University of New Mexico, July 2000.

\expar
Theory and Application of Self-reference.
PHILOG Conference, Denmark, October 2002.

\expar
The Metacognitive Loop.
CS and Electrical Computer Engineering Colloquium
University of Wyoming, August 2004

\expar
The Metacognitive Loop.
Institute for Cognitive Science Colloquium
University of Colorado, August 2004


\subsubsec ~~Unrefereed Conference Papers

\paper
A quick version of G\"{o}del's theorem.
AMS Meeting, New York, abstracted in
{\sl AMS Notices}, 
Feb. 1972.

\paper
Topics in the semantics of circumscription.
{\sl Week on Logic and Artificial Intelligence},
University of Maryland, October 1984.

\paper
Issues in commonsense reasoning.
{\sl Proceedings, Technical Symposium on Intelligent Systems}, 
(invited paper), ACM, June 1985.

\paper
Heidegger and artificial zoology.
(With A.~Kyburg.)
Annual conference of the {\sl Society for Philosophy
and Psychology}, 
April, 1989, Tucson, Arizona.

\paper
Virtual symposium on virtual mind.
(With P.~Hayes, S.~Harnad, and N.~Block.)
{\sl Minds and Machines}, vol.~2, pp.~217-238, 1992.

\eject

\subsec Grants

%\expar
%Martin Marietta Corporation, grant for summer Research Assistant, 1984.
%\expar
%National Science Foundation, (with Edmundson, Mills, Minker, and Smith), 
%\$10,695, 1984-1985.
\expar
Univ.~of Md. Foundation (from Martin Marietta Corp.), \$24,000, 1985.
\expar
Army Research Office, (with J.~Minker), \$100,600, 
1985-1986. %DAAG29-85-K-0177
\expar
Army Research Office, (with J.~Minker), \$110,460, 
1986-1987. %DAAG29-85-K-0177
\expar
Army Research Office, (with J.~Minker), \$121,100,
1987-1988. %DAAG29-85-K-0177
\expar
Army Research Office, (with J.~Minker), \$138,554,
1988-1989. %DAAL03-88-K0087
\expar
Army Research Office, (with J.~Minker), \$147,890,
1989-1990. %DAAL03-88-K0087
\expar
Army Research Office, (with J.~Minker), \$158,484,
1990-1991. %DAAL03-88-K0087
\expar
National Science Foundation, (with J.~Horty), \$42,032,
1990-1991. %IRI-8907122
\expar
National Science Foundation, (with J.~Horty), \$19,501,
1991. %supplement to IRI-8907122
\expar
Army Research Office, \$50,000, 1994-1995. %DAAH0494G0238
\expar
National Science Foundation (with S.~Kraus), \$200,000,
1994-1997. %IRI-9311988
\expar
Army Research Office (AASERT), \$70,000,
1995-1997. %DAAH049510274
\expar
National Science Foundation (equipment supplement), \$12,070.80, 1995.
\expar
Army Research Office, \$123,235, 1995-1997. %DAAH049510628 
\expar
National Science Foundation (postdoctoral associate supplement), \$27,750,
 1996-1997.
\expar
National Science Foundation (with S.~Kraus and M.~Morreau), \$100,000,
1997-1999.
\expar
Air Force Office of Scientific Research, \$490,960, 1999-2002.
\expar
Office of Naval Research and AFOSR, \$299,331, 2000-2003.
\expar
Office of Naval Research, \$300,00, 2002-2005.
\expar
Air Force Office of Scientific Research, \$520,000, 2002-2005.
\expar
Honda Research Institute, \$25,000, 2003-2004 (gift).
\expar
Honda Research Institute, \$105,653, 2004-2005.



\eject
\subsec Awards and Honors

\expar
Woodrow Wilson Foundation Fellowship, 1966
\expar
National Science Foundation Fellowship, 1966-69
\expar
General Research Board Summer Award, Univ.~of Maryland, 1983
\expar
University of Maryland Institute for Advanced Computer Studies,
research appointment, 1986-1992, 1993-2007.
\expar
Invited by Canadian Society for Computational Studies of
Intelligence, to give one-hour
lecture at the Canadian Artificial Intelligence conference, 
May 1992, Vancouver.
\expar
Invited by Society for Machines and Mentality, to give one-hour 
lecture at annual conference, December 1992, Washington D.C.
\expar
Journal paper 6 (above) listed in D.~Bobrow's 1993 compilation of the
most frequently-cited papers in {\sl Artificial Intelligence} from
1970 to 1991.)

\subsec Reviewing Activities for Journals and Other Scholarly
Institutions (last five years)

%\medskip

%I have served as referee or reviewer for the following scholarly
%publications, conferences, and institutions:

%\medskip

%\inpar {1984} {.75in}
%Reviewer, NSF Proposals.

%\inpar {1985} {.75in}
%Referee, IJCAI papers.

%\inpar {1985} {.75in}
%Referee, 
%{\sl Wiley Encyclopedia of Artificial Intelligence}.

%%\inpar {1985} {.75in}
%Referee,
%{\sl J. Phil. Logic}.

%\inpar {1985} {.75in}
%Referee, 
%{\sl Artificial Intelligence}.

%\inpar {1985} {.75in}
%Reviewer, NSF Proposal.

%\inpar {1986} {.75in}
%Referee,
%{\sl J. of Philosophical Logic}.

%\inpar {1986} {.75in}
%Referee,
%{\sl Computational Intelligence}.

%\inpar 1986 .75in
%Reviewer, NSF Proposal.

%\inpar 1986 .75in
%Referee, 
%{\sl Artificial Intelligence}.

%\inpar {1987} {.75in}
%Referee, IJCAI papers.

%\inpar {1987} {.75in}
%Referee,
%{\sl Computational Intelligence}.

%\inpar {1987} {.75in}
%Reviewer of
%{\sl Logical Foundations of AI}
%by  M.~Genesereth and N.~Nilsson.

%\inpar 1987 .75in
%Referee, 
%{\sl Artificial Intelligence}.

%\inpar 1987 .75in
%Referee, 
%{\sl International J. of Intelligent Systems}.

%\inpar 1987 .75in
%Referee, 
%{\sl J. of Systems and Software}.

%\inpar 1988 .75in
%Referee, 
%{\sl Artificial Intelligence}.

%\inpar 1988 .75in
%Reviewer, NSF Proposals.

%\inpar 1988 .75in
%Reviewer, ARO Proposals.

%\inpar 1988 .75in
%Referee,
%{\sl IEEE Transactions on Systems, Man, and Cybernetics}.

%\inpar 1988 .75in
%Referee,
%{\sl Computational Linguistics}.

%\inpar 1988 .75in
%Reviewer of book manuscript for MIT Press.

%\inpar 1988 .75in
%Referee,
%{\sl J. of Experimental and Theoretical Artificial Intelligence}.

%\inpar 1989 .75in
%Reviewer, ARO Proposal.

%\inpar 1989 .75in
%Referee,
%{\sl Computational Linguistics}.

%\inpar 1989 .75in
%Referee,
%{\sl International J. of Intelligent Systems}.

%\inpar 1989 .75in
%Reviewer, NSF Proposal.

%\inpar 1989 .75in
%Referee,
%{\sl Journal of Symbolic Computation}.

%\inpar 1989 .75in
%Tenure Review Committee for Penn. State Univ.~assistant professor.

%\inpar 1990 .75in
%Third Year Review Committee for New York Univ.~assistant professor.

%\inpar 1990 .75in
%Referee, ISMIS papers.

%\inpar 1990 .75in
%Referee, NACLP papers.

%\inpar 1990 .75in
%Referee,
%{\sl Journal of the Association for Computing Machinery}.

%\inpar 1991 .75in
%Referee, IJCAI-91 papers.

%\inpar 1991 .75in
%Referee, AAAI-91 papers.

%\inpar 1991 .75in
%Referee, ISMIS-91 papers.

%\inpar 1991 .75in
%Reviewer, NSF proposals.

%\inpar 1991 .75in
%Referee,
%{\sl Journal of the Association for Computing Machinery}.

%\inpar 1992 .75in
%Referee,
%{\sl Journal of the Association for Computing Machinery}.

%\inpar 1992 .75in
%Referee,
%{\sl Artificial Intelligence}.

%\inpar 1992 .75in
%Referee,
%{\sl Computational Intelligence}.

%\inpar 1992 .75in
%Referee, ISMIS-93 papers

%\inpar 1993 .75in
%Referee, IJCAI-93 papers.

%\inpar 1993 .75in
%Referee, BISFAI-93 papers.

%\inpar 1994 .75in
%Referee, TIME-94 papers.

%\inpar 1994 .75in
%Referee,
%{\sl Journal of the Association for Computing Machinery}.

%\inpar 1994 .75in
%Referee,
%{\sl Computational Intelligence}.

%\inpar 1994 .75in
%Referee,
%{\sl Journal of Artificial Intelligence Research}.

%\inpar 1994 .75in
%Reviewer, ARO proposal.

%\inpar 1995 .75in
%Referee, TIME-95 papers.

%\inpar 1995 .75in
%Referee,
%{\sl Computational Intelligence}.

%\inpar 1995 .75in
%Referee,
%{\sl Journal of Artificial Intelligence Research}.

%\inpar 1995 .75in
%Referee, {\sl J. of Experimental and Theoretical Artificial
%Intelligence}.

%\inpar 1995 .75in
%Referee, TIME-96 papers.

%\inpar 1996 .75in
%Reviewer, ARO proposals.

%\inpar 1996 .75in
%Reviewer,
%{\sl MIT Press}.

%\expar International Joint Conferences on Artificial Intelligence (IJCAI)
%\expar American Association for Artificial Intelligence (AAAI)
%\expar International Symposium on Methodologies for Intelligent Systems (ISMIS)
\expar National Science Foundation
\expar Army Research Office
%\expar {\sl Journal of the Association for Computing Machinery}
\expar {\sl Artificial Intelligence Journal}
\expar {\sl Computational Intelligence}
\expar {\sl Journal of Artificial Intelligence Research}
\expar {\sl Journal of Experimental and Theoretical Artificial Intelligence}
\expar {\sl MIT Press}
\expar {\sl AAAI-04}

%\eject
\sec Teaching and Advising

\subsec Courses Taught (last five years)

{\obeylines\parindent=0pt
%Spring 1983
%\quad CMSC 250: Introduction to Discrete Structures, 155 students
%\quad CMSC 828B: Knowledge Representation, 11 students
%\smallskip
%Fall 1983
%\quad CMSC 450: Elementary Logic and Algorithms, 117 students (2 secs.)
%\smallskip
%Spring 1984
%\quad CMSC 250: Introduction to Discrete Structures, 146 students
%\quad CMSC 828B: Functional Semantics, 17 students
%\smallskip
%Fall 1984
%\quad CMSC 450: Elementary Logic and Algorithms, 95 students (2 secs.)
%\smallskip
%Spring 1985
%\quad CMSC 250: Introduction to Discrete Structures, 95 students
%\quad CMSC 828B: Robots, Cognition, and Logic, 24 students
%\smallskip
%Fall 1985
%\quad CMSC 620: Problem Solving Methods in Artificial Intelligence, 55 students
%\quad CMSC 828E: Parallelism, Beliefs, and Default Reasoning, 20 students
%\smallskip
%Spring 1986
%\quad CMSC 620: Problem Solving Methods in Artificial Intelligence, 49 students
%\smallskip
%Fall 1986
%\quad no teaching duties: full-time research
%Spring 1987
%\quad CMSC 250: Introduction to Discrete Structures, 37 students
%\smallskip
%Fall 1987
%\quad CMSC 450: Elementary Logic and Algorithms, 41 students
%\smallskip
%Spring 1988
%\quad CMSC 452: Elementary Theory of Computation, 65 students
%\smallskip
%Fall 1988
%\quad no teaching duties: sabbatical leave
%\smallskip
%Spring 1989
%\quad no teaching duties: sabbatical leave
%\smallskip
%Fall 1989
%\quad CMSC 452: Elementary Theory of Computation, 39 students
%\quad CMSC 828P: Advanced Introduction to Cognitive Science, 10 students%
%%%\smallskip
%Spring 1990
%\quad no teaching duties: full-time research
%\smallskip
%Fall 1990
%\quad no teaching duties: full-time research
%\smallskip
%Spring 1991
%\quad CMSC 150: Introduction to Discrete Structures, 85 students
%\quad CMSC 280: Computer Science II, 40 students
%\smallskip
%Fall 1991
%\quad on leave of absence, University of Akron
%Spring 1992
%\quad on leave of absence, University of Akron
%\smallskip
%Fall 1992
%\quad no teaching duties: full-time research
%\smallskip
%Spring 1993
%\quad CMSC 150: Introduction to Discrete Structures, 152 students (2 sections)
%\quad CMSC 450: Logic for Computer Science, 27 students
%\smallskip
%Fall 1993
%\quad no teaching duties: full-time research
%\smallskip
%Spring 1994
%\quad CMSC 150: Introduction to Discrete Structures, 31 students
%\quad CMSC 828P: Advanced Introduction to Cognitive Science, 7 students
%\smallskip
%Fall 1994
%\quad CMSC 105: Pascal Programming, 32 students
%\smallskip
%Spring 1995
%\quad CMSC 450: Logic for Computer Science, 30 students
%\smallskip
%Fall 1995
%\quad CMSC 105: Pascal Programming, 35 students
%\smallskip
%Spring 1996
%\quad CMSC 721: Nonmonotonic Reasoning, 6 students
%\smallskip
%Fall 1996
%\quad UNIV 128A: Origins, 200 students
%\smallskip
%Spring 1997 (planned)
%\quad CMSC 421: Introduction to Artificial Intelligence

% \smallskip
% 1992-93
% \quad CMSC 150: Introduction to Discrete Structures, 152 students (2 sections)
% \quad CMSC 450: Logic for Computer Science, 27 students
% \smallskip
% 1993-94
% \quad CMSC 150: Introduction to Discrete Structures, 31 students
% \quad CMSC 828P: Advanced Introduction to Cognitive Science, 7 students
% \smallskip
% 1994-95
% \quad CMSC 105: Pascal Programming, 32 students
% \quad CMSC 450: Logic for Computer Science, 30 students
% \smallskip
% 1995-96
% \quad CMSC 105: Pascal Programming, 35 students
% \quad CMSC 721: Nonmonotonic Reasoning, 6 students
% \smallskip
% 1996-97
% \quad UNIV 128A: Origins, 200 students (and 25-student recitation section)
% \quad CMSC 421: Introduction to Artificial Intelligence (40 students)
% \smallskip
% 1997-98
% \quad CMSC 421: Introduction to Artificial Intelligence (51 students)
% \quad CMSC 150: Introduction to Discrete Structures (105 students)

\smallskip
1998-99
\quad CMSC 251: Introduction to Algorithms (2 sections, $\sim 55$ students each)
\smallskip
1999-00
\quad CMSC 251: Introduction to Algorithms (2 sections, $\sim 55$ students each)
\smallskip
2000-01
\quad CMSC 251: Introduction to Algorithms (2 sections, $\sim 50$ students each)
\smallskip
2001-02
\quad GEMS 496 and 497: Gemstone Seminar (8 students; with B. Dorr)
\quad CMSC 297: Honors Seminar (11 students; with B. Dorr, W. Gasarch)
\quad CMSC 251: Introduction to Algorithms (2 sections, $\sim 90$ students each)
\smallskip
2002-2003
\quad GEMS 296 and 297: Gemstone Seminar (9 students)
\quad CMSC 351: Introduction to Algorithms (2 sections, $\sim 80$ students each)
\smallskip
2003-2004
\quad GEMS 396 and 397: Gemstone Seminar (8 students)
\quad CMSC 297: Honors Seminar (11 students; with B. Dorr, W. Gasarch)
\quad CMSC 351: Introduction to Algorithms (2 sections, $\sim 80$ students each)
\quad CMSC 838: How to do research (with B. Dorr, W. Gasarch)
\smallskip
2004-2005
\quad GEMS 496: Gemstone Seminar (8 students)
\quad CMSC 838: How to do research (with B. Dorr, W. Gasarch)
\smallskip
2005-2006
\quad CMSC 351: Introduction to Algorithms (2 sections)
\quad GEMS 497: Gemstone Seminar (8 students)
\quad CMSC 838: How to do research (with C. Kruskal, W. Gasarch)
}

%\eject
\subsec Curriculum, Notes, and Other Contributions to Teaching

\paper
Algebraic topology:  an introduction to modern analytic structures with
mention of physical applications, 
Lecture  notes,  University  of  Puerto
Rico, 1977.

\paper
{\sl Lecture notes in logic} 
(with P.~Steitz, M.~Miller, and C.~Laskowski), 1995.

\paper
Development of graduate course CMSC 721 (Nonmonotonic Reasoning),
with Jack Minker, Jeff Horty, VS Subrahmanian, and Michael Morreau.


\subsec Advising

\bigskip

\subsubsec ~~Other Than Research Direction

\medskip
\noindent Approximately 5 undergraduate and 3 graduate students
are advised each year.

During 1999-2000, jointly with Bob Anderson in the UMCP Physics Dept,
I advised Dan Giambalvo on an honors project in quantum
computing.  I subsequently
advised Pete Schwartz on an honors project in reasoning with
contradictory information.


\bigskip

\subsubsec ~~Research Direction

\bigskip
\noindent
UMCP Summer Research Opportunity Program mentor for Yolanda
Pena-Vargas, 1995.

\medskip
%{\obeylines\parindent=0pt
\noindent
{\sl Master's}
\smallskip
\noindent
Dumar Villa 1977, James Abello 1978, Ruben Pereira 1978, Krishnan Subramanian 1986 (with J.~Minker),
Diana Gordon 1986,
Hachidai Ito 1987,
Tom Melton 1988,
Sanjoy Paul 1988,
Will Sterbenz 1989 (with D.~Kueker),
Michael Miller 1989,
Michael Harris 1991 (with S.~Kraus)
%}

%\eject

{\obeylines\parindent=0pt
\medskip
{\sl Doctoral}
\smallskip
Jennifer Elgot-Drapkin 1988
~~~{\sl Step-logic: Reasoning situated in time.}
~~~Assistant Professor, Arizona State University, Tempe AZ (1988-97)
\smallskip
Diana Gordon 1990 
~~~{\sl Active bias adjustment for incremental, supervised concept learning.}
~~~Associate Professor, University of Wyoming.
\smallskip
Michael Miller 1993
~~~{\sl A view of one's past and other aspects of reasoned change in belief.}
~~~Computer Scientist, Aether Systems, Inc.
\smallskip
Frank McFadden 1993 (with J.~Reggia and H.~Szu)
~~~{\sl Competitive learning and competitive activation in cortical map formation.}
~~~Computer Scientist, General Dynamics
\smallskip
Madhura Nirkhe 1994 (with J.~JaJa)
~~~{\sl Time-situated reasoning within tight deadlines and realistic
~~~space and computation bounds.}
~~~Computer Scientist, Microsoft Corporation
\smallskip
Khemdut Purang 2001
~~~{\sl Systems that detect and repair their own mistakes.}
~~~Computer Scientist, Sony Corporation.
\eject
{\sl Current Doctoral: }
\bigskip


Darsana P. Josyula (candidacy in 2003; PhD expected in Spring 2005)
\medskip
Waiyian Chong (candidacy in 2003; PhD expected in Spring 2005)

}

\bigskip
\bigskip
%\eject
{\obeylines\parindent=0pt
{\sl Faculty Research Associates}
\bigskip
Dr.~Sarit Kraus 1988-90
~~~Professor, Bar Ilan University, Israel
~~~Recipient of Computers and Thought Award, 1995.
~~~Elected Fellow of AAAI, 2002.
\medskip
Dr.~Elizabeth Klipple 1995-97
~~~Army Research Laboratory
\medskip
Dr.~David Traum 1996-2000
~~~Institute for Creative Technologies, University of Southern California
\medskip
Dr.~Michael Anderson 2001-
\medskip
Dr.~Ken Hennacy 2003-
}


\sec Service

\subsec Professional

\inpar {1984} {.75in}
A non-monotonic reasoning bibliography,
{\sl Proceedings, Workshop on Non-monotonic Reasoning}, 
AAAI, New Paltz, 1984.

\inpar {1984-5} {.75in}
Co-organizer, Week on Logic and Artificial Intelligence, October 1984, as
part of special year in Logic and Theoretical Computer Science.

\inpar {1989-90} {.75in}
Co-organizer, Annual Meeting of the Society for Philosophy and Psychology.

\inpar {1990} {.75in}
Program Committee, Fifth International Symposium on Methodologies for
Intelligent Systems.

\inpar {1990-} {.75in}
Organizer, {\sl Society for Philosophy and Psychology} Electronic Mailing List.

\inpar {1990-} {.75in}
Editorial Board, {\sl Journal of Logic and Computation}.

\inpar {1991-} {.75in}
Editorial Board, {\sl Computational Intelligence}.

\inpar {1991-} {.75in}
Editorial Board, {\sl Fundamenta Informaticae}.

\inpar {1997-} {.75in}
Editorial Board (de facto member) and Book Review Editor,
{\sl Artificial Intelligence}.

\inpar {1991} {.75in}
Program Committee, Sixth International Symposium on Methodologies for
Intelligent Systems.

\inpar {1991} {.75in}
Program Committee, National Conference on Artificial Intelligence (AAAI-1991).

\inpar {1993} {.75in}
Program Committee, BISFAI-93 (Third Bar-Ilan Symposium on Foundations of AI).

\inpar {1993} {.75in}
Participant in external review of Artificial Intelligence Program,
University of Georgia.

\inpar {1993-94} {.75in}
Program Committee, TIME-94 (Workshop on temporal representation and
reasoning, to be held in conjunction with the 1994 Florida Artificial
Intelligence Research Symposium.

\inpar {1993-94} {.75in}
Program Committee (Workshop Program Chair), 1994 National Conference
on Artificial Intelligence (AAAI-94), Seattle.

\inpar {1994-95} {.75in}
Program Committee, Workshop on Rational Agency: Concepts, Theories,
Models, and Applications. AAAI Fall Symposium Series. MIT, 1995.

\inpar {1994-95} {.75in}
Program Committee, TIME-95.

\inpar {1995} {.75in}
Participant in Army Research Office/TRADOC Spoken Human-Machine
Dialogue Workshop, North Carolina, May 1995.

\inpar {1995-96} {.75in}
Program Committee, TIME-96.

\inpar {1995-98} {.75in}
Executive Committee, Society for Philosophy and Psychology.
%\eject

\inpar {2000} {.75in}
NSF Review Panel.

\inpar {2003-04} {.75in}
Program Committee, DARPA Workshop on Self-Aware Systems.

\inpar {2004} {.75in}
Program Committee, 
National Conference on Artificial Intelligence (AAAI-04)

\inpar {2005} {.75in}
Program Committee, Nineteenth International Joint Conference on
Artificial Intelligence

\inpar {2005} {.75in}
Program Committee, CommonSense 2005, the Seventh Symposium on Logical
Formalizations of Commonsense Reasoning

\inpar {2005} {.75in}
Program Committee, First International Workshop on Formal Models of
Resource Bounded Agents

\subsec University

\subsubsec ~~Departmental

\inpar {1982-} {.75in}
Member, Information Processing Field Committee.
\inpar {1982-90} {.75in}
Member, Theory Field Committee.
\inpar {1982-84} {.75in}
Organizer, weekly AI discussion group (with Nau and Weiser)
\inpar {1983} {.75in}
Colloquium Host for Pat Hayes.
\inpar {1983} {.75in}
Theory Representative, August Comprehensive Exam Committee.
\inpar {1983-4} {.75in}
Member, Committee to revise the Comprehensive Exam Structure.
\inpar {1983-4} {.75in}
Organizer, Bi-Weekly Discussion Group on Artificial Intelligence.
\inpar {1985} {.75in}
Member, Committee to revise 200-level courses.
\inpar {1985} {.75in}
Member, Committee on Technical Reports.
\inpar {1985} {.75in}
Theory Representative, January Comprehensive Exam Committee.
\inpar {1986} {.75in}
Colloquium Host for Hector Levesque.
\inpar {1986} {.75in}
Member, Ad Hoc Committee to organize Faculty Retreat.
\inpar {1986-8} {.75in}
Organizer, Departmental Weekly On-Line Bulletin of Events.
\inpar {1986-7} {.75in}
Chair, AI Field Subcommittee.
\inpar {1987} {.75in}
Colloquium Host for Drew McDermott.
\inpar {1987-8} {.75in}
Chair, Committee for High School Day '88.
\inpar {1987-8} {.75in}
Liaison for Mathematics, Linguistics, Philosophy, and Cognitive Studies.
\inpar {1989-90} {.75in}
IP Comprehensive Examination Coordinator.
\inpar {1989-90} {.75in}
Chair, IP Field Committee.
\inpar {1989-90} {.75in}
Faculty advisor of student ACM chapter.
\inpar {1990-91} {.75in}
Member, New Building Committee.
\inpar {1990-91} {.75in}
Chair, Ad Hoc Committee to Re-evaluate the Comprehensive Exam Structure
\inpar {1992} {.75in}
Committee on Professional Master's Degree.
\inpar {1992} {.75in}
AI Comprehensive Examination Coordinator.
\inpar {1993-95} {.75in}
Graduate Studies Committee, advisor on minority students
\inpar {1993-94} {.75in}
Dept Representative, College Diversity Committee
\inpar {1994-95} {.75in}
Dept Retention/Graduation Coordinator, Office of Academic Affairs
\inpar {1994-95} {.75in}
Ad Hoc Committee on Instructor Tenure Criteria
\inpar {1995-96} {.75in}
Graduate Admissions Committee
\inpar {1995} {.75in}
Ad Hoc Committee on Communication Skills
\inpar{1995-96} {.75in}
Chair, AI Field Committee
\inpar{1995-96} {.75in}
Member, Academic Evaluation Committee
\inpar{1995-97} {.75in}
Chair, Colloquium Series
\inpar{1997-98} {.75in}
General Editor, CSD Annual Report
\inpar {2000-01} {.75in}
Chair, AI Field Committee.
\inpar {2000-} {.75in}
Co-Chair, Honors Program
\inpar {2000-} {.75in}
Senate Representative
\inpar {2002} {.75in}
Ad Hoc Committee on Publicity
\inpar {2003-} {.75in}
Create/maintain web page for CS Undergraduate Research
\inpar {2004} {.75in}
Graduate Admissions Committee
\inpar {2004} {.75in}
Teaching Evaluation Committee
\inpar {2004-7} {.75in}
Chair, AI Field Committee

\subsubsec ~~College

\inpar {1983-4} {.75in}
Member, organizing committee for the special year in Logic and Theoretical
Computer Science, Mathematics Department.
\inpar {1984-8} {.75in}
Co-organizer, Joint Robot Project, with Departments of Computer Science,
Electrical Engineering, Mechanical Engineering, and the Center for
Automation Research.
\inpar {1986} {.75in}
Member, Ad Hoc Hearing Board on Academic Dishonesty.
\inpar {1989-} {.75in}
Member, Committee for Cognitive Studies
\inpar {1992-3} {.75in}
Member, College Space Committee
\inpar {1993-4} {.75in}
Demo organizer, AI Day (UMIACS, CSD, CFAR)
\inpar {1995-} {.75in}
Co-organizer, Logic and AI Seminar (UMIACS)
\inpar {2003-4} {.75in}
Ad Hoc Committee on Electronic FAR (Faculty Activity Report)

\subsubsec ~~Campus and University

\inpar {1983} {.75in}
Department Representative, United Way Campaign
\inpar {1984-5} {.75in}
Co-organizer, Study Group on Cognitive Science.
\inpar {1986-7} {.75in}
Equal Employment Opportunity Officer, UMIACS.
\inpar {1988} {.75in}
Member, Minority Recruiting Committee, UMIACS.
\inpar {1990-97} {.75in}
Co-organizer, UMCP Cognitive Science Electronic Mailing List.
\inpar {1990-91} {.75in}
Department Representative, Faculty Senate
\inpar {1990} {.75in}
Department Liaison and Steering Committee, Neuroscience Center
\inpar {1993} {.75in}
Participant in external review of UMCP Philosophy Department
\inpar {1993-6} {.75in}
Member, General Research Board Committee
\inpar {1997} {.75in}
Participant in external review of UMCP Linguistics Department
\inpar {1999-} {.75in}
Gemstone mentor
\inpar {2003-} {.75in}
Member, UMCP Graduate Council
\inpar {2004-} {.75in}
Member, PCC for Graduate Council
\inpar {2004-} {.75in}
Member, NACS Curriculum Committee

\subsec External

\inpar {1995} {.75in}
Member of College Park Ad Hoc Working Group for Town-Center and
Metro Planning

\inpar {1995-6} {.75in}
Planning and execution of seminars for high school seniors at Prince
William County Schools, especially with regard to computer science and
cognitive science, and also in debate competitions.

\inpar {1997} {.75in}
Mentor for Westinghouse Competition project by Mr.~Jason Ernst of
Montgomery Blair High School. His project was awarded a semifinalist
position nationwide, one of 300 such out of over 1500 projects entered.

\inpar {2000} {.75in}
Science Fair Judge, University Park Elementary School.

\inpar {2003-} {.75in}
Assistant Scoutmaster, Boy Scout Troop 214.
\bye

