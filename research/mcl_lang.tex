%\documentclass{amsart}
\documentclass[fullpage]{article}

\textwidth = 400pt
\oddsidemargin = 25pt
\evensidemargin = 25pt

\title{\bf The MetaCognitive Loop and Language Learning\\
	--- A Position Statement}

\author{Don Perlis\\
        University of Maryland\\
        College Park MD 20742 USA\\
        perlis@cs.umd.edu\\
        www.umd.edu/~perlis\\
        301-405-2685
        fax: 301-405-6707}


\begin{document}
\maketitle

\begin{abstract}

In our current work (see http://www.cs.umd.edu/active) 
we postulate a "metacognitive loop" in both human and machine
commonsense reasoning, that allows humans (and should allow
machines) to function effectively in novel situations, by noting
 errors and adopting strategies for dealing with them.  The loop
 has three main steps: (i) monitor events for a possible anomaly,
 (ii) assess its type and possible stategies for dealing with it,
 and (iii) guide one or more strategies into place while continuing
 to monitor (looping back to step i) for new anomalies that may
 arise either as part of the strategy underway or otherwise.

 One domain that appears very natural for application of this loop
 is natural language human-computer dialog.  Errors of miscommunication
 are prevalent in dialog, especially human-computer dialog.  Our work
 to date has indicated that something analogous to our loop not only
 is active in human dialog but also can be a powerful tool for automated
 systems.  One example is the learning of new words: if a word is used
 that the system does not know, this can be processed as an anomaly,
 which in turn can trigger various stategies including asking for help
 ("What does that word mean?").  Other examples include disambiguations,
 and appropriate retraction of implicatures and presuppositions.  One
 aim is a computer system that, via dialog with humans, learns
 a much larger vocabulary in the process of noting and correcting its
 misunderstandings, akin to a foreigner learning a new language.  A
 more ambitious goal is that of learning a new grammar (and/or
 correcting an  ``old'' grammar.)

We have already had some success in automated (i) identification of
mismatches between existing lexical information and the ongoing course
of a given dialog; (ii) asking for clarification in such cases; (iii)
one-off learning of clarificational definitions.


\end{abstract}

%\begin{verbatim}




%\end{verbatim}

\bibliographystyle{plain}
\bibliography{/fs/disco/group/bibfiles/ALL,/fs/disco/group/bibfiles/perlis1}



\end{document}
