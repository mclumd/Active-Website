%% LyX 1.1 created this file.  For more info, see http://www.lyx.org/.
\documentclass[]{llncs}
\usepackage{amsmath}
\usepackage{times}


\begin{document}

\title{Seven Days in the Life of a Robotic Agent:\\
\small{Bootstrapping Intelligence by Continual Reflection}}


\author{Waiyian Chong\inst{1} \and Mike O'Donovan-Anderson\inst{2} \and
  Yoshi Okamoto\inst{2}\inst{3} \and Don Perlis\inst{1}\inst{2}\\
  \{yuan,mikeoda,yoshi,perlis\}@cs.umd.edu}

\institute{Department of Computer Science, University of Maryland, College
  Park, MD 20742 \and Institute for Advanced Computer Studies, University
  of Maryland \and Department of Linguistics, University of Maryland}

\maketitle


%\section*{Outline (Draft)}
%
%\begin{itemize}
%\item End: Self-improving behavior (theoretical and practical (creative
%laziness) motivations);
%\item Means: reflection and continual computation;
%\item Reflection: taking advantage of reasoning capability to improve the
%implementation of the reasoning system, completing a benign, mutual
%reinforcing cycle; providing a path to automatic compilation of procedural
%knowledge and specialized modules;
%\item Continual computation: reflection is expensive; need to take
%advantage of idle time to do precomputation (supporting point:
%no-free-lunch theorem);
%\item Logic based system is an (the most?) appropriate medium (supporting
%points: uniformity (lemma as universal format for caching
%computation\cite{bibel1997:let_plan}), transparency(incremental
%introduction of more detailed knowledge when necessary),
%expressiveness(higher cognitive capacities, BDI, interestingness, etc for
%free));
%\item Active logic: role of time in the feeling of being bored? Relations
%among time and interestingness and memory and history?
%\item Related works: progress made in solving the frame problem (Reiter,
%Shanahan), success in declarative specification of heuristics in planning
%(Bacchus and Kabanza), Cognitive Robotics by Toronto Gang (Levesque et al),
%UTexas (Baral?), etc).
%\end{itemize}
%\pagebreak

%%%%%%%%%%%%%%%%%%%%%%%%%%%%%%%%%%%%%%%%%%%%%%%%%%%%%%%%%%%%%%%%%%%%%%%%
%%% Body starts here:-

\begin{abstract}
  Bootstrapping is a widely employed technique in the process of building
  highly complex systems such as microprocessors, language compilers, and
  computer operating systems.  It could play an even more prominent role in
  the creation of computation systems capable of supporting intelligent
  agent behaviors because of the even higher level of complexity.  The
  prospect of a self-bootstrapping, self-improving intelligent system has
  motivated various fields of research in machine learning.  However, a
  robust, generalizable methodology of machine learning is yet to be found;
  there are still a lot of learning behaviors that no existing learning
  technique can adequately account for.  We believe a uniform, logic-based
  system such as active logic
  \cite{elgot-drapkin/perlis:reasoning:jetai,elgot-drapkin/kraus/miller/nirkhe/perlis1996:active_logics}
  (see Sec~\ref{activelogic} below for more details), will be more
  successful in the realization of this ideal.  The overall architecture
  that we envision is as follows: a central commonsense reasoner module
  attends to novel situations where the system does not already have
  expertise, and to its own failures; it then reasons its way to solutions
  or repairs, and puts these into action while at the same time causing
  {}``expert'' modules to be either created or retrained so as to more
  quickly enact those solutions on future occasions.  Thus what we propose
  is a kind of meld between declarative and procedural techniques where the
  former has great expressive power and flexibility (but is slow) and the
  latter is very fast but hard to adapt to new situations.  We will explore
  the possibilities of using reflection and continual computation toward
  this end.
\end{abstract}

\section{Introduction}

A unifying theme of AI research is the design of an architecture for
allowing an intelligent agent to operate in a \emph{common sense informatic
situation} \cite{mccarthy1989:artificial_intelligence}, where the agent's
perception (hence its knowledge about the world) is incomplete, uncertain
and subjected to change; and the effect of its actions indeterministic and
unreliable.  There are many reasons (e.g., scientific, philosophical,
practical) to study intelligent agent architecture; for our purposes, we
will define our goal as to improve the performance of the agent, where
performance is in turn defined as resources (time, energy, etc) spent in
completing given tasks.  We are interested in the question {}``What is the
best strategy to build an agent which can perform competently in a common
sense informatic situation?''  It is clear that it will be impractical for
the designer of the agent to anticipate everything it may encounter in such
a situation; hence it is essential that some routes of self-improvement be
provided for the agent if it is to attain reasonable level of autonomy.
What should we provide to the architecture to open these routes?

For an agent to function competently in commonsense world, we can expect
the underlying architecture to be highly complex.  Careful attention should
be paid to the designing process, as well as the designed artifact to
ensure success.  We identified the following requirements to guide our
design: (i) In addition to fine-tuning of specialized modules the agent
might have, more fundamental aspects of the architecture should be open to
self-improvement.  For example, an agent designed to interact with people
may have a face recognition module; a learning algorithm to improve its
face recognition accuracy is of course desirable, but it is not likely to
be helpful for the agent to cope with unexpected changes in the world.
(ii) Improvements need to be made reasonably efficiently.  It's said that a
roomful of monkeys typing away diligently at their keyboards will
eventually produce the complete works of Shakespeare; in the same vein, we
can imagine a genetic algorithm, given enough time and input, can evolve a
sentient being, but the time it takes will likely be too long for us to
withstand.  (iii) Somewhat related to the previous two points, it is
important to stress that the improvements made be transparent to us so that
we can incrementally provide more detailed knowledge and guidance when
necessary to speed up the improvements.

%The prospect of a self-bootstrapping, self-improving intelligent system has
%motivated various fields of research in machine learning.  However, a
%robust, generalizable methodology of machine learning is yet to be found;
%there are still a lot of learning behaviors that no existing learning
%technique can adequately account for.  We believe a uniform, logic-based
%system such as active logic
%\cite{elgot-drapkin/perlis:reasoning:jetai,elgot-drapkin/kraus/miller/nirkhe/perlis1996:active_logics}
%(see Sec~\ref{activelogic} below for more details), will be more successful
%in the realization of this ideal.
  
Toward building an intelligent agent, we can borrow a few lessons from
builders of other sophisticated systems: in particular, the technique of
bootstrapping is of relevance.  Bootstrapping technique has been widely
employed in the process of building highly complex systems such as
microprocessors, language compilers, and computer operating systems.  It
could play an even more prominent role in the creation of computation
systems capable of supporting intelligent agent behaviors, because of the
even higher level of complexity.  Typically in a bootstrapping process, a
lower-level infrastructural system is first built ``by hand''; the complete
system is then built, within the system itself, utilizing the more powerful
constructs provided by the infrastructure.  Hence, it provides benefits in
the ways of saving effort as well as managing complexity.
%There are several reasons for stressing bootstrapping.  Practically, as
%alluded to above, it is an instance of effort saving {}``constructive
%laziness'':

Ideally, as designers of the agent, we'd like to push as much work as
possible to be automated and carried out by computer.  There is no doubt
that the study of specialized algorithms has been making great
contributions to the realization of intelligent agency; however, we think
that the study of bootstrapping behavior may be a more economical way to
achieve that goal.  Instead of designing the specialized modules ourselves,
we should instead look for way to provide the infrastructure on which
agents can discover and devise the modules themselves.

%Scientifically, we are interested in the following questions: If we leave
%alone a robot agent in a reasonably rich environment for a long period of
%time, what will enable the robot to evolve itself into a more competitive
%agent?  How do we provide a path for the agent to improve itself?  We think
%self-monitoring, self-modifying, and ultimately self-improving capabilities
%are essential components of any intelligent agent; and reflection provides
%a principled mechanism to realize this capabilities.  The possession of
%these capacities by an agent provides it with competitive advantage for
%survival (for a biological agent) or for better completing its design goals
%(for an artificial agent) in an unpredictable, changing environment.

Once we accept bootstrapping as a reasonable way to proceed, a few natural
questions arise: What constructs are needed in the infrastructure to
support the bootstrapping of intelligence?  How should they be combined?
How should they operate?  More generally, if we leave alone a robot agent
in a reasonably rich environment for a long period of time, what will
enable the robot to evolve itself into a more competitive agent?  How do we
provide a path for the agent to improve itself?  We think an example will
help us to answer the questions!  In the next section, we will tell the
story of Al the office robot, to show the importance and desirability of
self-improving capability in an artificial agent.  In light of the typical
problems that a robot may encounter in the real world, the following two
sections (Sec~\ref{reflection} and Sec~\ref{continual}) present a more
detailed account of two key ideas: reflection and continual computation,
which we think are essential to the success of the robot, and argue that
the uniformity and expressiveness of a logic-based system can facilitate
the implementation of complex agency.  In section \ref{activelogic}, we
will give a brief introduction to Active Logic, the theoretical base of our
implementation, and discuss why we think this approach is promising.  It is
nonetheless clear that there are still very many problems to solve before
Al can be more than science fiction.


\section{Seven Days in the Life of Al}

Let us consider Al, a robot powered by active logic.  Al is an {}``office
robot'', who roams the CS office building delivering documents, coffee,
etc.  Al was endowed at birth with the desire to make people happy.  We
will see how Al developed into an excellent office robot of great
efficiency through its first week of work.

\noindent \textbf{1st day}: Al was first given a tour of the building.
Among other things, it was shown the power outlets scattered around the
building so that it could recharge itself.

\noindent \textbf{2nd day}: The morning went well: Al delivered everything
on target.  But during the afternoon Al ran into a problem: it found itself
unable to move! The problem was soon diagnosed --- it was simply a low
battery.  (Since thinking draws less energy than moving, Al could still
think.) It turned out that although Al knew it needed power to operate and
it could recharge itself to restore its battery, it had never occurred to
Al that, {}``it would need to reach an outlet
before the power went too low for it to move!'' %
\footnote{Counter to traditional supposition that all derivable formulas
  are already present in the system.  } The movement failure triggered Al
to derive the above conclusion, but it was too late; Al was stuck, and
could not deliver coffee on request.  Caffeine deprived computer scientists
are not happy human beings; Al had a bad day.

\noindent \textbf{3rd day}: Al was bailed out of the predicament by
its supervisor in the morning.  Having learned its lesson, Al decided to
find an outlet a few minutes before the battery got too low.  Unfortunately
for Al, optimal route planning for robot navigation is an NP-complete
problem.  When Al finally found an optimal path to the nearest power
outlet, its battery level was well below what it needed to move, and Al was
stuck again.  Since there was nothing else it could do, Al decided to surf
the web (through the wireless network!), and came upon an interesting
article titled {}``Deadline-Coupled Real-time Planning''
\cite{nirkhe/kraus/miller/perlis:how}.

\noindent \textbf{4th day}: After reading the paper, Al understood
that planning takes time, and that it couldn't afford to find an optimal
plan when its action is time critical.  Al decided to quickly pick the
outlet in sight when its battery was low.  Unfortunately, the outlet
happened to be too far away, and Al ran out of power again before reaching
it.  In fact, there was a closer outlet just around the corner; but since a
non-optimal algorithm was used, Al missed it.  Again, stuck with nothing
else to do, Al kicked into the {}``meditation'' mode where it called the
Automated Discovery (AD) module to draw new conclusions based on the facts
it accumulated these few days.  Al made some interesting discoveries: upon
inspecting the history of its observations and reasonings, Al found that
there were only a few places it frequented; it could actually precompute
the optimal routes from those places to the nearest outlets.  Al spent all
night computing those routes.

Meanwhile, Al also built a special-purpose (procedural) navigator module NM
to navigate to those routes, so that (i) the navigation would be faster and
(ii) it could spend more time attending to other matters such as sorting
mail for delivery (this being a more error-prone task requiring commonsense
analysis).

\noindent \textbf{5th day}: This morning, Al's AD module derived an
interesting theorem: {}``if the battery power level is above 97\% of
capacity when Al starts (and nothing bad happened along the way), it can
reach an outlet before the power is exhausted.'' Al didn't get stuck that
day.  But people found Al to be not very responsive.  Later, it was found
that Al spent most of its time around the outlets recharging itself ---
since Al's power level dropped 3\% for every 10 minutes, the theorem above
led it to conclude that it needed to go to the outlet every 10 minutes.

It also turned out that two of the power outlets it used for recharging
became inoperative and AL had to deliberately (reason its way to) override
its navigator module NM, and retrain it to avoid those outlets.

\noindent \textbf{6th day}: After Al's routine introspection before
work, it was revealed that the knowledge base was populated with millions
of theorems similar to the one it found the day before, but with the power
level at 11\%, 12\%, ..., and so on.  In fact, the theorem is true when the
power level is above 10\% of capacity.  Luckily, there was a meta-rule in
Al's knowledge base saying that {}``a theorem subsumed by another is less
interesting;'' thus all the theorems with parameter above 10\% were
discarded.  Equipped with this newer, more accurate information, Al
concluded that it could get away with recharging itself every 5 hours.

\noindent \textbf{7th day}: That happened to be Sunday.  Nobody was
coming to the office.  Al spent its day contemplating the meaning of life.


Analyzing the behavior of Al, we can see a few mechanisms at play: in
addition to the basic deductive reasoning, goal directed behavior, etc., Al
also demonstrates capabilities such as abductive reasoning (diagnoses of
failures), explanation-based learning (compilation of navigation rules,
derivation of recharging rules), reflection (examining and reasoning about
its power reading, revision of recharging rule), and time-sensitivity
(understanding that deliberations take time, people don't like waiting,
etc).  Of course, none of these is new in itself; however, the interactions
among them has enabled Al to demonstrate remarkable flexibility and
adaptivity in a ill-anticipated (by the designer of Al) and changing world.
Below, we will elaborate on the reflective capability and the continual
aspect of the agent's operations.
%It is interesting to note...

%\subsection{Random Thoughts {[}To be eliminated!{]}}
%
%\begin{itemize}
%\item Paradoxically, the path to efficiency is not by limiting the
%  expressive power of the representation language; but increasing it.
%\item Justified to step back and demand higher level of discourse:
%  imperative to production-rules to horn-clauses to fol to hol to
%  metalogic.  Increase the autonomy of agents, give them more
%  responsibilities, let them improve themselves.  Tuning for performance (by
%  devising specialized representations and algorithms (strips planning?))
%  is not the job that human researchers should have; it will be made
%  obsolete in a few years by advance in computer hardware.  The way to go is
%  to go meta: think of way to make this performance tuning happens
%  automatically.
%\item Learning fundamentally is reflection: it is about the changes in the
%  configuration of the system to improve its performance.  Reflection
%  completes a mutual reinforcing circle; it provides competitive advantage
%  to the agent that utitilizes it; it can speed up the learning process.
%\item Explicit preferable to implicit: goals are implicit in production
%  rules systems, they are not first class citizens; it is inconvenient to
%  express, thus reason about goals in such systems.
%\end{itemize}


\section{Reflection\label{reflection}}

%Traditionally, researchers have focussed their attention on solving
%individual, well-specified problems (or on specifing a problem in the
%simplest most computationally tractable way).  For example, in a typical
%learning task, a set of training examples are given, together with certain
%measure of success (e.g., minimizing error).  It has been the job of the
%researchers to identify the problems underlying learning, and decide which
%are the most interesting and worth solving.  When considering such matters,
%it is often beneficial to take a step back, and see whether one is on the
%right track, or whether there are better strategies to solve the problem.
%Obviously, this requires metareasoning or {\em reflection}, and the case of
%problem identification and strategy selection seems to require
%meta-reasoning of an extremely advanced sort.  Our contention is,
%therefore, that any attempt to automate the general process of problem
%identification and solution must be rooted in extremely robust reflective
%capacities.

A computational system is said to be reflective when it is itself part of
its own domain (and in a causally connected way).  More precisely, this
implies that (i) the system has an internal representation of itself, and
(ii) the system can engage in both {}``normal'' computation about the
external domain and {}``reflective'' computation about itself
\cite{maes1988:computational_reflection}.  Hence, reflection can provide a
principled mechanism for the system to modify itself in a profound way.

We suggest that a useful strategy for a self-improving system is to use
reflection in the service of self-training.  Just as a human agent might
deliberately practice a useful task, increasing her efficiency until (as we
say) it can be done ``unconsciously'' or ``automatically'', without
explicit reasoning, we think that once a reflective system identifies an
algorithm or other method for solving a frequently encountered problem, it
should be able to create procedural modules to implement the chosen
strategy, so as to be able in the future to accomplish its task(s) more
efficiently, without fully engaging its (slow and expensive) common-sense
reasoning abilities.

%Of course, all the capabilities mentioned above are in principle within
%reach of the usual learning algorithms.  To appreciate the benefits of
%reflection, we can understand a reflective system as follows:
%
%We can understand a computation system as consisted of two components: the
%formal reasoning system (R), and the mechanical computation system (M)
%implementing the formal reasoning system.  

%A long-standing issue in the research of agent architecture is the tension
%between deliberative and reactive control.  With few exceptions (e.g.,
%\cite{brooks1991:intelligence_without}), people seem to accept that an
%agent need both to function well in real world.  The prevailing designs so
%far usually feature different layers of control, where higher-level layers
%are responsible for long term deliberation and lower layers responsible for
%quick reaction
%\cite{lesperance1994:logical_approach,bonasso1996:using_layered}.  However,
%this clear separation of levels creates a barrier and limits the
%flexibility of the architecture.  Reflection seems to have the potential to
%overcome this.

Although reflection sounds attractive, it has largely been ignored by
researchers of agent architecture, mainly because of the high computation
complexity involved in doing reflective reasoning.  However, we think the
solution to the problem is not by avoiding reflection, but looking at the
larger picture and considering the environment and extent in which an agent
operates, and finding way to reap the benefits of reflection without being
bogged down by its cost.  We think the notion of continual computation is a
promising venue for reflection to become useful.

\section{Continual Computation\label{continual}}

Any newcomer to the field of AI will soon find out that, almost without
exception, all {}``interesting'' problems are NP-hard.  When a computer
scientist is confronted with a hard problem, there are several options to
deal with it.  For example, one can simplify the problem by assuming it
occurs only under certain conditions (which are not always realistic) and
hoping bad cases don't happen frequently.  One can also identify a simpler
subproblem so that it can be solved algorithmically and automated, and
leave the hard part for the human.  Another option is for the scientist to
study the problem carefully, derive some heuristics, and hope that they
will be adequate most of the time.  But none of these is quite satisfying:
ideally, we would like the computer to do as much work for us as possible,
and hopefully, be able to derive the heuristics by itself.  A promising
approach toward realizing this ideal is the notion of \emph{continual
  computation} \cite{horvitz1997:models_continual}.

The main motivation behind continual computation is to exploit the
\emph{idle time} of a computation system.  As exemplified by usage patterns
of desktop computers, workstations, web-servers, etc. of today, most
computer systems are under utilized: in typical employments of these
systems, relatively long spans of inactivity are interrupted with bursts of
computation intensive tasks, where the systems are taxed to their limits.
How can we make use of the idle time to help improve performance during
critical time?

Continual computation generalizes the definition of a \emph{problem} to
encompass the uncertain stream of challenges faced over time.  One way to
analyze this problem is to put it into the framework of probability and
utility, or more generally, rational decision making:

\begin{quotation}
  Policies for guiding the precomputation and caching of complete or
  partial solutions of potential future problems are targeted at enhancing
  the expected value of future behavior.  The policies can be harnessed to
  allocate periods of time traditionally viewed as idle time between
  problems, as well as to consider the value of redirecting resources that
  might typically be allocated to solving a definite, current problem to
  the precomputation of responses to potential future challenges under
  uncertainty\cite{horvitz2001:principles}.
\end{quotation}
An implicit assumption of the utility-based work in continual computation
is that the future is somehow predictable.  But in many cases, this cannot
be expected.  For example, for long term planning, most statistics will
probably lose their significance.  Here is a place where logic-based
systems with the capability to derive or discover theorems on its own
(e.g., Lenat's AM system) can play a complementary role, similar to the way
that mathematics plays a complementary role to engineering.  Just as
mathematicians usually do not rely on immediate reward to guide their
research (yet discover theorems of utmost utility), AM can function in a
way independent of the immediate utility of its work.

More precisely, if we adopt logic as our base for computation and look at
problem solving as theorem proving \cite{bibel1997:let_plan}, a system
capable of discovering new theorems can become a very attractive model of a
continual computation system.  In such a system, every newly discovered
theorem has the potential of simplifying the proof of future theorem; so in
essence, theorems become our universal format for caching the results of
precomputation and partial solutions to problems.

A simplistic embodiment of the model can just be a forward chaining system
capable of combining facts in its database to produce new theorems using
modus ponens, for instance.  Such a system is not likely to be very useful,
however, because it will spend most of its time deriving uninteresting
theorems.  So the success of this model of continual computation will hinge
on whether we can find meaningful criteria for the {}``interestingness'' of
a theorem.  In the classical AM
\cite{lenat1982:am,lenat1983:theory_formation,lenat/brown1984:why_am}, the
system relies largely on human judgment determine interestingness.  In a
survey of several automated discovery programs, Colton and Bundy
\cite{colton/bundy1999:notion_interestingness} identify several properties
of concepts which seem to be relevant to their interestingness, such as
novelty, surprisingness, understandability, existence of models and
possibly true conjectures about them.  Although these properties seem
plausible, it is not obvious they are precise enough to be operational to
guide automated discovery programs toward significant results.


\section{Toward Implementation\label{activelogic}}

%{[}Description of Active Logic; extensions to support metareasoning,
%reflection, etc.{]}
%
%Looking back at our motivating scenario, several features of \emph{active
%  logic} helped Al: its time sensitivity enabled Al to realized that
%optimality is not necessarily desirable, especially when there is a
%deadline approaching.  More fundamentally, the active logic's treatment of
%time allowed Al to keep track of changes: the battery level is \( X \) now
%does not imply that is still true 5 hours later; active logic's \(
%\mathit{Now}(i) \) predicate provides a natural and efficient way to deal
%with that.  The history mechanism in active logic gave Al an opportunity to
%spot certain pattern in its past behavior, which helped it improve its
%future behavior.  Because of the uniformity of the logic-based system,
%precomputation is seamlessly integrated into goal based problem solving
%through the forward- and backward-chaining mechanisms.  Finally, the
%expressiveness of active logic made it possible to store the meta-rules
%about interestingness of theorems, which gave Al good {}``taste'' in
%evaluating them.
%
%{[}Future works...{]}
Considering for a moment a few general requirements for a real-world
agent like Al, it is obvious, first of all, that Al must perceive its
environment.  Further, if it is to reason about what he perceives, and
use this reasoning to guide its actions, it must be capable of coming
to have perceptually grounded empirical beliefs.\footnote{For a simple
example of why perceptions are not enough, consider that when Al needs
to charge itself, it may not be able, at that moment, to perceive a
charging station.  In this case it will need to {\em remember} where
(the nearest) one is; thus even this basic action requires both
perception (that his battery is low) and belief (that there is a
station around the corner).}  But the world is always changing, and if
Al's beliefs are going to be useful, they would best reflect
the world as it actually is--and that means that Al must be capable
of forming new beliefs, and revising or getting rid of old beliefs, in
response to real-time perceptual input from the world.  Al's belief
system, that is to say, must be appropriately reactive.

But it is clear that reactivity is not enough.  For in any reasoning
system, where new beliefs are constantly being derived from old ones,
reactivity introduces a difficult problem: if a belief which is an
antecedent condition in a sequence of reasoning itself changes, then
the consequents of that sequence may be invalidated.  Al must be able
to recognize when this happens, and take appropriate steps to solve
the problem.  Thus, Al needs to be a perceiver, a believer, a
reasoner, and a meta-reasoner, capable of reasoning not just {\em
with} its beliefs, but {\em about} them.  Further (and crucial to the
possibility of boot-strapping), Al is concerned not just with meeting
its various first-order goals (delivering coffee, recharging) in light
of the changing environment in which it acts, but also in fulfilling
second-order goals \cite{Frankfurt:Importance} -- goals about what
sort of agent to become -- such as increased efficiency.
Meta-reasoning in light of these second-order desires is a somewhat
complicated and delicate task, for it can involve not just deciding
how to do something, but deciding how to decide to do it, as when Al
builds a special navigator module to compute paths to the various
charging stations.  Al, this is to say, must be both reactive and
reflective, attending not just to the external world, but also to its
own mind, finding ways to bring about its goals and desires in both
spheres.

Al, therefore, must be able to recognize and react to changes in the
environment, and to deal with the contradictions, irregularities, and
invalidated conclusions that this will involve; further, in choosing when
and how to act, it must be able to consider not just how best to achieve a
given goal in light of the state of the world, but also which goals are the
best to achieve, and which methods of achieving them are best given the
sort of agent it wants to become.\footnote{Interestingly, Aristotle's
notion of character -- roughly, a stable disposition for choosing which goals
to achieve and which methods to employ in so doing -- is more useful in
understanding an agent like Al than are more modern notions of choice and
agency, in which a radically free act of the will plays a larger role.  For
it is clear that the more decisions Al makes about what sort of agent to
become -- the more specialized modules and simple, reactive, procedural
systems it builds for dealing with everyday, common situations -- the more
ossified its reactions to the world will become, and the more difficult it
will be to change.  Balancing the need for spontaneity against that for
stability and efficiency is at the core of agency, both human and
artificial.  For more on the tensions of agency see,
e.g. \cite{Bratman_Faces,Bratman/87:inten_plans_pract}; for more on
Aristotelian notions of character
\cite{Aristotle:Ethics,Broadie:Aristotle}.}  Together, this suggests the
need for extremely rich representations of Al's beliefs, goals, and
desires, sufficient to support robust meta-reasoning.  What is needed,
then, for the reasoning engine of a real-world agent, is something
reactive, flexible and expressive, a reasoning system in which beliefs can be
added, changed, and removed in real-time without disrupting the reasoning
process, and which can support introspection and meta-reasoning.  Active
Logic was designed, and is being continually developed, with these
desiderata in mind.

As is detailed in, e.g.,
\cite{Elgot-etalTR,elgot-drapkin/perlis:reasoning:jetai,Purang-prac99}
active logic is one of a family of inference engines (step-logics)
that explicitly reason in time, and incorporate a history of their
reasoning as they run.  Motivated in part by the thought that human
reasoning takes place step-wise, in time--and that this feature
supports human mental flexibility--Active Logic works by combining
inference rules with a constantly evolving measure of time (a "Now")
that can itself be referenced in those rules. As an example, from
$\mathit{Now}(t)$--the time is now "$t$"--one infers $\mathit{Now}(t+1)$, for the fact of an
inference implies that time (at least one 'time-step') has passed.
All the inference rules in Active Logic work temporally in this way:
at each time-step all possible one step inferences are made, and only
propositions derived at time $t$ are available for inferences at time
$t+1$.  There are special persistence rules so that every theorem
$\alpha$ present at time $t$ implies itself at time $t+1$; likewise there
are special rules so that if the knowledge base contains both a
theorem $\alpha$ and its negation $\neg \alpha$, these theorems and
their consequences are "distrusted" so they are neither carried
forward themselves nor used in further inference.  These features,
along with a quotation mechanism allowing theorems to refer to each
other, give active logic the expressive and inferential power to
monitor its own reasoning in a real-time fashion, as that very
reasoning is going on, allowing it to watch for errors (such as
mismatches between the environment and expectations), to note temporal
aspects of actions or reasoning (an approaching deadline, or that
progress is or is not occurring) which might dictate the adoption of a
different goal or strategy, and to exert reasoned control over its
past and upcoming inferential processes, including re-examination and
alteration of beliefs and inferences.  Active logic therefore supports
flexible reasoning, and is well suited for complex and ever-changing
real-world contexts like autonomous agency and human-computer dialog.

A simple example of active logic inference is shown, below.
\begin{align*}
i:   & \mathit{Now}(i),   A, A \rightarrow B \\
i+1: & \mathit{Now}(i+1), A, A \rightarrow B, B
\end{align*}
 
%\smallskip 
%\enumsentence{\tt 
%\begin{tabular}{r@{ : }lll} 
%  i& Now (i), & A,& A -> B\\ 
%i+1& Now(i+1),& A,& A -> B, B\\ 
%\end{tabular}} 
 
%\smallskip 
\noindent Here $i$ and $i+1$ in the left margin indicate time steps, and
the propositions to the right are (some of) the beliefs in the KB at those
times. Among the latter are beliefs of the form $\mathit{Now}(t)$, i.e.,
the logic ``knows'' what time it is. This time-stratified knowledge
representation and reasoning is crucial to many applications of active
logics, from deadline-coupled planning
\cite{nirkhe/kraus/miller/perlis:how}, to time-sensitive inference in
general \cite {elgot-drapkin/perlis:reasoning:jetai}, to
contradiction-detection \cite {miller/perlis:presentations}, to
discourse pragmatics \cite {gurney/perlis/purang:interpreting,Traum_ETAI}
(allowing the introduction of temporal subtleties into certain formal
treatments of presupposition such as \cite {heim:presupposition}).
 
In the example above, at step $i+1$, $B$ has just been inferred from $A$
and $A \rightarrow B$ at step $i$.  Also illustrated is the fact that
beliefs at one step need not be ``inherited'' to the next, e.g.,
$\mathit{Now}(i)$ is not inherited to step $i+1$. This disinheritance
feature can also be applied to other beliefs, and is important in dealing
with contradictory beliefs. When contradictions are encountered, active
logic can, first of all, disinherit the contradictands so they do not cause
further untrustworthy beliefs, and second, retrace its history of
inferences to examine what led to the contradiction, performing
metareasoning concerning which of these warrants continued belief
\cite{miller/perlis:presentations,gurney/perlis/purang:interpreting}.  (For
certain domains in which the automated system is a helper or advice-taker,
it can also simply pass along the contradictory situation to a human user
and await advice.) Indeed, a primary aim of our research into active logics
has been to explore the extent to which contradictions may be categorized
and generic strategies found that successfully resolve particular
contradiction types.  The central point for now, however, is that our use
of active logic is based on its expressive power, and its ability to
support the kinds of metareasoning we appear to need for a fully situated
agency.

In addition to providing support for the desiderata mentioned
above--flexible, non-monotonic, real-time reasoning--active logic's
time-sensitivity helped Al in other ways. For instance, the fact that
Al recognized that its reasoning itself took time allowed it to
realize that optimality is not necessarily desirable, especially when
there is a deadline approaching. More fundamentally, the active logic
treatment of time allowed Al to keep track of changes: that the battery
level is \( X \) now does not imply that this is still true 5 hours later;
active logic's \( \mathit{Now}(i) \) predicate provides a natural and 
efficient way to deal with both reasoning {\em in} time, and also 
reasoning {\em about}
time.  Further, the history mechanism in active logic gave Al an opportunity
to spot certain pattern in its past behavior, which helped it improve
its future behavior (or, speaking more generally, the
time-situatedness of Al's reasoning provides a natural framework to capture
the future-orientation of agency, and the past-orientation required
by learning).

Finally, because of the uniformity of a logic-based system,
precomputation is seamlessly integrated into goal based problem
solving through forward- and backward-chaining reasoning
mechanisms. And the expressiveness of active logic made it possible to
store the meta-rules about such things as interestingness of theorems,
or the value and meaning of efficiency, which gave Al the basis
for certain kinds of bootstrapping.  These features allow for learning
behavior without traditional learning mechanisms (explanation base
learning, inductive learning, etc.) being explicitly programmed.

%\section{Related Works}
%
%Researchers have identified a set of capabilities which seem to be
%essential for an embeded agent to function effectively: for example, we
%need \emph{reactivity} to response quickly to unanticipated events;
%we need the ability for \emph{deliberation} to avoid wasted effort%
%\footnote{By modeling the world and simulation of possible courses of
%  actions, a deliberative agent needs not actually carry out the actions,
%  which is usually much more expensive than deliberation, to see the
%  consequences; and thus can make more sensible decision than a purely
%  reactive agent.  } and optimize performance.  Impatience with the initial
%seeming ineffectiveness of the logical approach
%\cite{green1969:application_theorem}, researchers devise specialized
%algorithms and formalisms \cite{nilsson:principles} to get acceptable
%results.
%
%{[}...{]}
%
%{[}Progress made in solving the frame problem (Reiter, Shanahan), success
%in declarative specification of heuristics in planning (Bacchus and
%Kabanza), Cognitive Robotics by Toronto Gang (Levesque et al), UTexas
%(Baral?), etc).  {]}


\section{Conclusions}

The so called \emph{No Free Lunch Theorem}
\cite{wolpert/macready1997:no_free} states that {}``all algorithms that
search for an extremum of a cost function perform exactly the same, when
averaged over all possible cost functions.'' In other words, without domain
specific structural assumptions of the problem, no algorithm can be
expected to perform better on average than simple blind search.  This
result appears to be a cause for pessimism for researchers hoping to devise
domain-independent methods to improve problem solving performance.  But on
the other hand, this theorem also provides compelling reason for embracing
the notion of continual computation, which can be seen as a way to exploit
domain dependent information in a domain independent way.

However, to take advantage of continual computation, we need to be able to
express the concept of interestingness in a way sufficient to guide
computation to profitable direction.  Interestingness touches on the
ultimate uncertainty: what to do next?  Although utility theory has its
place, we argued that there are aspects of interestingness not susceptible
to utility based analysis.  We believe that a forward and backward chaining
capable logic system such as active logic, with its expressiveness, time
sensitivity, and reflective ability to reason about theorems, proofs and
derivations, is well-positioned to take advantage of the opportunity
offered by continual computation, and can serve as a solid basis for the
realization of intelligent agency.

%%% Body ends here.
%%%%%%%%%%%%%%%%%%%%%%%%%%%%%%%%%%%%%%%%%%%%%%%%%%%%%%%%%%%%%%%%%%%%%%%%


\bibliographystyle{splncs}

\bibliography{/fs/disco/group/bibfiles/ALL,/fs/disco/group/bibfiles/Mikeoda}

\end{document}
